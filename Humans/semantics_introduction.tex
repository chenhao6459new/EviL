In this section I turn to developing the formal semantics for
\tmtextsc{EviL} with a single agent.  I imagine the object of study in
\textsc{EviL} is an agent, which I call the \textsc{EviL} agent.  In
\S\ref{multi-agent}, the semantic framework offered here is extended to
incorporate multiple agents. In Appendix \ref{alternative}, yet another
framework is offered employing gamelike semantics, which avoids the
grammar restriction suggested in \S\ref{sketch}.

The grammar restriction imposed on \textsc{EviL} was introduced to
avoid paradoxes. That being the case, I shall discard the previous
definition of $(\VDash)$ I suggested, in favor of demonstrably well-defined semantics.  This shall be achieved in two steps.

\begin{definition} Let $\mathcal{L}_0 (\Phi)$ be the language of classical propositional logic, defined by the following Backus-Naur form grammar:
\[ \phi \ {::=} \  p \in \Phi \  | \  \phi
   \rightarrow \psi \  | \  \bot \]
\end{definition}
Models for classical propositional logic can be thought of as sets $S \subseteq \Phi$; thus the truth predicate $(\models) \colons \powerset \Phi \rightarrow \mathcal{L}_0 (\Phi)
\rightarrow \textup{\textsf{bool}}${\footnote{$\ldots$ where $\textup{\textsf{bool}} := \{True, False\}$.  This is more commonly written as
$(\models) \colons \powerset \Phi \rightarrow
\textup{\textsf{bool}}^{\mathcal{L}_0 (\Phi)}$.  My notation reflects
the notation common to the typed functional programming languages
\tmtextit{Haskell} and \tmtextit{OCaml}. I will use both notations interchangeably.}} for classical propositional logic
can be given recursively as follows:  
\begin{definition}
Define $(\models)$ such that
\begin{align*}
  S{\models}p & {\iff}p{\in}S\\
  S{\models}{\phi}{\rightarrow}{\psi} & {\iff}S{\models}{\phi}\text{ implies
  }S{\models}{\psi}\\
  S{\models}{\bot} & {\iff} False
\end{align*}
\end{definition}
Further, observe that the language $\mathcal{L}_0$ is extended by \tmtextsc{EviL}
\begin{definition} Define $\mathcal{L} (\Phi)$ by the following Backus-Naur grammar:
\[ \phi \ {::=} \  p \in \Phi \  | \  \phi
   \rightarrow \psi \  | \  \bot \  |
   \  \Box \phi \  | \  \boxminus \phi
   \  | \  \boxplus \phi \  | \ 
   \circlearrowleft \]
\end{definition}
 \tmtextsc{EviL} models are sets $\Omega
\subseteq \powerset \Phi \times \powerset
\mathcal{L}_0 (\Phi)$.  Like classical propositional logic, semantics for
\tmtextsc{EviL} are given recursively by a predicate
\[ (\VDash) \colons \powerset  (\powerset \Phi \times \powerset \mathcal{L}_0 (\Phi)) \rightarrow \powerset \Phi
   \times \powerset \mathcal{L}_0 (\Phi)
   \rightarrow \mathcal{L} (\Phi) \rightarrow
   \textup{\textsf{bool}}. \]
That is, $(\VDash)$ is a function that:
\begin{itemizedot}
  \item Takes as input:
\begin{itemizedot}
  \item An \evil model
  
  \item A pair $(a, A)$ where
  \begin{itemizedot}
    \item $a\subseteq \Phi$ is a set of proposition letters
    
    \item $A\subseteq \mathcal{L}_0 (\Phi)$ is a set of propositional formulae.
  \end{itemizedot}
  \item A formula in the language $\mathcal{L} (\Phi)$
  \end{itemizedot}
  \item Gives as output: a truth value in $\textup{\textsf{bool}}$
\end{itemizedot}
I can now provide a formal definition of the semantics for \textsc{EviL}:
%{\footnote{Where $X \in \mathcal{A}$, I use $A$ to denote $A (X)$ provided that $A \colons \mathcal{A} \rightarrow\powerset \mathcal{L}_0 (\Phi)$}}
\begin{definition}
 Define $(\VDash)$ recursively such that:
\begin{align*}
  {\Omega},(a,A){\VDash} p & {\iff}p{\in}a\\
  {\Omega},(a,A){\VDash} {\phi}{\rightarrow}{\psi} &
  {\iff}{\Omega},(a,A){\VDash}{\phi}\text{ implies
  }{\Omega},(a,A){\VDash}{\psi}\\
  {\Omega},(a,A){\VDash}{\bot} & {\iff} False\\
  {\Omega},(a,A){\VDash}\Box {\phi} & {\iff}{\forall}(b,B){\in}{\Omega}.
  ({\forall}{\psi}{\in}A. b{\models}{\psi})\text{ implies
  }{\Omega},(b,B){\VDash}{\phi}\\
  {\Omega},(a,A){\VDash}{\boxminus}{\phi} &
  {\iff}{\forall}(b,B){\in}{\Omega}. a=b\text{ and }B{\subseteq}A\text{
  implies }{\Omega},(b,B){\VDash}{\phi}\\
  {\Omega},(a,A){\VDash}{\boxplus}{\phi} &
  {\iff}{\forall}(b,B){\in}{\Omega}. a=b\text{ and }B{\supseteq}A\text{
  implies }{\Omega},(b,B){\VDash}{\phi}\\
  {\Omega},(a,A){\VDash}{\circlearrowleft} & {\iff}
  {\forall}{\psi}{\in}A.a{\models}{\psi}
\end{align*}
I will write $\Omega \VDash \phi$ to mean $\Omega, (a,A) \VDash \phi$ for all $(a,A) \in \Omega$.  Further, I will write $\VDash \phi$ to mean $\Omega \VDash \phi$ for all $\Omega$.
\end{definition}
These semantics are well defined, since apart from relying on the semantics
for propositional logic they may be observed to be compositional.{\footnote{In
fact, I have provided a formulation of these semantics in the same manner as
above in the computer proof assistant Isabelle/HOL \citep{nipkow_isabelle/hol:proof_2002}; I shall
give my remarks on formal verification in \S\ref{formal}.  In the case of
Isabelle/HOL, the function $(\VDash)$ was defined inductively, and
automatically proven to be well-defined.  Specifically, the conditions given
above specify that $(\VDash)$ is determined by a \tmtextit{monotonic}
predicate over suitable tupples, and similarly for $(\models)$.  Hence the
result that $(\VDash)$ is well-defined ultimately relies on an application of
the \tmtextit{Knaster-Tarski Fixpoint Theorem} \citep[chapter 12]{roman_lattices_2008}. Further,
since I have given an inductive definition, these recursive definitions rely
on the {\tmem{least}} fixpoint of their associated monotonic operators.}} 
Moreover, the following relationship can be observed:

\begin{lemma}[Truthiness]\label{truthiness}
  Let $\phi \in \mathcal{L}_0 (\Phi)$.  Then:
  \[ a \models \phi \Longleftrightarrow \Omega, (a, A) \VDash \phi \]
  $\ldots$for any $\Omega$ and $A$.
\end{lemma}
\begin{proof}
  This may be seen immediately by induction on $\phi$.
\end{proof}

$\ldots$with this, we have the following, mirroring Prop. \ref{central-prop}:
 
\begin{definition}  Define the following:
 $$Th(\Omega) := \{ \phi \in \mathcal{L}(\Phi) \ |\ \Omega \VDash \phi \}$$
\end{definition}

\begin{theorem}[Theorem Theorem]\label{theorem-theorem}
  If $A$ is finite, then $\Omega, (a,A) \VDash \Box \phi$ if and only if $Th(\Omega) \cup A \vdash_{\text{\textsc{EviL}}} \phi$.
\end{theorem}

I shall present $\vdash_{\text{\textsc{EviL}}}$, the logical consequence turnstile for \textsc{EviL}, in \S\ref{evil-axioms}.

I chose the name ``Theorem Theorem'' because it means that for every belief the \textsc{EviL} agent has, it is a theorem she has derived from her premises. Theorem \ref{theorem-theorem} establishes one of the central desiderata outlined in \S\ref{close} is achieved by \textsc{EviL}.  With this result the foundation is set for the the central intuition driving \textsc{EviL} - that beliefs are the consequences of logical deductions.  It is a peculiarity of \textsc{EviL} that these deductions are carried on in \textsc{EviL} itself.  This was achieved, primarily, by my previous flirtation with paradox.  And as a consequence, I have tried to design \textsc{EviL} to eat its own tail. This is my favorite kind of self-reference. As a modeler using logic, it establishes that the \textsc{EviL} agent is herself also a modeler just like me, using the same logic I am using to think about her herself, to think about the state space she lives in.  It reminds me of a quote regarding the wonderful circularity of mathematics, due to \citet{browne_garden_1736}:
\begin{quote}
All things began in order, so shall they end, and so shall they begin again; according to the ordainer of order and mystical Mathematicks of the City of Heaven.\end{quote}


%%% Local Variables: 
%%% mode: latex
%%% TeX-master: "evil_philosophy"
%%% End: 
