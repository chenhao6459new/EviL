In this section we turn to developing the formal semantics for
\tmtextsc{EviL} with a single agent.  We shall imagine the object of study in
\textsc{EviL} is an agent, which we shall call the \textsc{EviL} agent.  In
\S\ref{multi-agent}, the semantic framework offered here is extended to
incorporate multiple agents. In Appendix \ref{alternative}, yet another
framework is offered employing gamelike semantics, which avoids the
grammar restriction suggested in \S\ref{sketch}.

The grammar restriction imposed on \textsc{EviL} was introduced to
avoid paradoxes. That being the case, we shall discard the previous
definition of $(\VDash)$ that was suggested in \S\ref{sketch}, in
favor of demonstrably well-defined semantics.  
This shall be achieved in two steps.

\begin{definition} Let $\mathcal{L}_0 (\Phi)$ be the language of
  classical propositional logic, defined by the following Backus-Naur form grammar:
\[ \phi \ {::=} \  p \in \Phi \  | \  \phi
   \rightarrow \psi \  | \  \bot \]
\end{definition}
Models for classical propositional logic can be thought of as sets $S
\subseteq \Phi$;
thus the truth predicate
%{\footnote{Here we are assuming
%    $\textup{\textsf{bool}} := \{True, False\}$.  
% This is more commonly written as
% $(\models) \colons \powerset \Phi \rightarrow (\mathcal{L}_0 (\Phi)
% \rightarrow \textup{\textsf{bool}})$ or $(\models) \colons \powerset \Phi \rightarrow
% \textup{\textsf{bool}}^{\mathcal{L}_0 (\Phi)}$.  The notation I
% present assumes that $\to$ associates to the left, following the
% convention for implication in logic. In general, my notation follows
% the typed functional programming languages \tmtextit{Haskell} 
% and \tmtextit{OCaml}.}
%} 
$(\models) \colons \powerset \Phi \times \mathcal{L}_0 (\Phi)
\rightarrow \textup{\textsf{bool}}$ for classical 
propositional logic can be given recursively as follows:
\begin{definition}
Define $(\models)$ such that
\begin{align*}
  S{\models}p & {\iff}p{\in}S\\
  S{\models}{\phi}{\rightarrow}{\psi} & {\iff}S{\models}{\phi}\text{ implies
  }S{\models}{\psi}\\
  S{\models}{\bot} & {\iff} False
\end{align*}
\end{definition}
Further, observe that the language $\mathcal{L}_0$ is extended by \tmtextsc{EviL}
\begin{definition} Define $\mathcal{L} (\Phi)$ by the following Backus-Naur grammar:
\[ \phi \ {::=} \  p \in \Phi \  | \  \phi
   \rightarrow \psi \  | \  \bot \  |
   \  \Box \phi \  | \  \boxminus \phi
   \  | \  \boxplus \phi \  | \ 
   \circlearrowleft \]
\end{definition}
Unlike traditional modal logic, \textsc{EviL} employs concrete models
rather than Kripke structures. \tmtextsc{EviL} models are sets 
$\Omega \subseteq \powerset \Phi \times \powerset \mathcal{L}_0
(\Phi)$.  Like classical propositional logic, semantics for
\tmtextsc{EviL} are given recursively by a predicate $(\VDash)$ which:
\begin{itemizedot}
  \item Takes as input:
\begin{itemizedot}
  \item An \evil model
  \item A pair $(a, A)$ where
  \begin{itemizedot}
    \item $a\subseteq \Phi$ is a set of proposition letters
    \item $A\subseteq \mathcal{L}_0 (\Phi)$ is a set of propositional formulae.
  \end{itemizedot}
  \item A formula in the language $\mathcal{L} (\Phi)$
  \end{itemizedot}
  \item Gives as output: a truth value in $\textup{\textsf{bool}}$
\end{itemizedot}

More concisely, this may be written as 
\[ (\VDash) \colons \powerset  (\powerset \Phi \times \powerset \mathcal{L}_0 (\Phi)) \times (\powerset \Phi
   \times \powerset \mathcal{L}_0 (\Phi)) \times \mathcal{L} (\Phi) \rightarrow
   \textup{\textsf{bool}}. \]

% The following provides a  of the semantics for 
% \textsc{EviL}:

\begin{definition}
 Define $(\VDash)$ recursively such that:
\begin{align*}
  {\Omega},(a,A){\VDash} p & {\iff}p{\in}a\\
  {\Omega},(a,A){\VDash} {\phi}{\rightarrow}{\psi} &
  {\iff}{\Omega},(a,A){\VDash}{\phi}\text{ implies
  }{\Omega},(a,A){\VDash}{\psi}\\
  {\Omega},(a,A){\VDash}{\bot} & {\iff} False\\
  {\Omega},(a,A){\VDash}\Box {\phi} & {\iff}{\forall}(b,B){\in}{\Omega}.
  ({\forall}{\psi}{\in}A. b{\models}{\psi})\text{ implies
  }{\Omega},(b,B){\VDash}{\phi}\\
  {\Omega},(a,A){\VDash}{\boxminus}{\phi} &
  {\iff}{\forall}(b,B){\in}{\Omega}. a=b\text{ and }B{\subseteq}A\text{
  implies }{\Omega},(b,B){\VDash}{\phi}\\
  {\Omega},(a,A){\VDash}{\boxplus}{\phi} &
  {\iff}{\forall}(b,B){\in}{\Omega}. a=b\text{ and }B{\supseteq}A\text{
  implies }{\Omega},(b,B){\VDash}{\phi}\\
  {\Omega},(a,A){\VDash}{\circlearrowleft} & {\iff}
  {\forall}{\psi}{\in}A.a{\models}{\psi}
\end{align*}
\end{definition}
\begin{remark}
We will write $\Omega \VDash \phi$ to mean $\Omega, (a,A) \VDash \phi$
for all $(a,A) \in \Omega$.  Further, we will write $\VDash \phi$ to
mean $\Omega \VDash \phi$ for all $\Omega$.
\end{remark}
These semantics are well defined, since apart from relying on the semantics
for propositional logic they may be observed to be compositional.
Moreover, the following relationship can be observed:

\begin{lemma}[Truthiness]\label{truthiness}
  Let $\phi \in \mathcal{L}_0 (\Phi)$.  Then:
  \[ a \models \phi \Longleftrightarrow \Omega, (a, A) \VDash \phi \]
  for any $\Omega$ and $A$.
\end{lemma}
\begin{proof}
  This may be seen immediately by induction on $\phi$.
\end{proof}

With this, we have the following, mirroring Prop. \ref{central-prop}:
\begin{definition}  Define the following:
 $$Th(\Omega) := \{ \phi \in \mathcal{L}(\Phi) \ |\ \Omega \VDash \phi \}$$
\end{definition}

\begin{theorem}[Theorem Theorem]\label{theorem-theorem}
  If $A$ is finite, then $\Omega, (a,A) \VDash \Box \phi$ if and only if $Th(\Omega) \cup A \vdash_{\text{\textsc{EviL}}} \phi$.
\end{theorem}
\begin{proof}
  The proof proceeds the exactly as the proof of Proposition
  \ref{central-prop} from \S\ref{sketch}.
\end{proof}

I shall present $\vdash_{\text{\textsc{EviL}}}$, the logical consequence turnstile for \textsc{EviL}, in \S\ref{evil-axioms}.

I chose the name ``Theorem Theorem'' because it means that for every
belief the \textsc{EviL} agent has, it is a theorem she has derived
from her premises. Theorem \ref{theorem-theorem} establishes one of
the central desiderata outlined in \S\ref{close} is achieved by
\textsc{EviL}.  With this result the foundation is set for the the
central intuition driving \textsc{EviL} - that beliefs are the
consequences of logical deductions.  It is a peculiarity of
\textsc{EviL} that these deductions are carried on in \textsc{EviL}
itself.  This was achieved, primarily, by flirting heavily with
paradox, as was illustrated in \S\ref{sketch}.  
As a consequence, we have tried to design \textsc{EviL}
to eat its own tail. 
It establishes that the \textsc{EviL} agent is
herself also a modeler just like us, using the same logic we are using
to think about her herself, to think about the state space she lives
in.

This sort of self referential circularity is a celebrated theme in
mathematics. It is similar, in a way, to the old alchemical conception
of mathematics, exemplified by the following quote 
due to Sir Thomas Browne:
\begin{quote}
All things began in order, so shall they end, and so shall they begin
again; according to the ordainer of order and mystical Mathematicks of
the City of Heaven.\citet[chapter 5]{browne_garden_1736}\end{quote}
Another, related notion of self reference was championed by Douglas
Hofstadter in his book \emph{G\"odel, Escher, Bach: An Eternal Golden
  Braid}, in what he calls ``Strange Loops'':
\begin{quote}
The flexibility of intelligence comes from the enormous number of
different rules, and levels of rules. The reason that so many rules on
so many different levels must exist is that in life, a creature is
faced with millions of situations of completely different types. In
some situations, there are stereotyped responses which require "just
plain" rules. Some situations are mixtures of stereotyped
situations-thus they require rules for deciding which of the ``just
plain'' rules to apply. Some situations cannot be classified-thus
there must exist rules for inventing new rules \ldots and on and
on. Without doubt, Strange Loops involving rules that change
themselves, directly or indirectly, are at the core of
intelligence. Sometimes the complexity of our minds seems so
overwhelming that one feels that there can be no solution to the
problem of understanding intelligence-that it is wrong to think that rules of any sort govern a creature's behavior, even if one takes ``rule''
in the multilevel sense described above. \cite[pg. 24]{hofstadter_godel_1979}
\end{quote}
In the setting of \textsc{EviL}, the strange loop is as follows: the
rules of the logic affect, indirectly, the truths of the semantics,
which in turn affect the rules of the logic.  It is perhaps dangerous
to engage in this sort of modeling, since it invites paradox.
However, the central goal of epistemic logic is to
provide a logic which models agents with knowledge.  This is because we are
ourselves agents with knowledge; epistemic logic's ultimate purpose is
to say ourselves by studying the subject.  \textsc{EviL} was designed
with to make true the Theorem Theorem, precisely with this final
reason in mind.

It cannot be stressed enough, Theorem \ref{theorem-theorem} is central
to the conceptual backing behind \textsc{EviL}.  
It provides the conceptual backbone of the perspective on 
epistemic logic this essay is intended to investigate.

%%% Local Variables: 
%%% mode: latex
%%% TeX-master: "evil_philosophy"
%%% End: 