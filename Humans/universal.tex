In this section, we show how \textsc{EviL} may be extended with a
universal modality.  Just as the previous section illustrated that
looking at fragments of \textsc{EviL} added complexity to the
completeness theorem, so too do natural extensions to the calculus.

In this section, we shall provide sketches rather than elaborate
proofs as we have so far provided.  This is because we really intend
for the results in this section to be minor modifications of our
previous results.  Our intention is to indicate what modifications are
to be made to accommodate our proposed extension.

The following gives the extended grammar of \textsc{EviL} with an added
universal modality:
$\mathcal{L}^U (\Phi,\mathcal{A})$ is the fragment:
\[ \phi \ {::=} \  p \in \Phi \  | \  \phi
   \rightarrow \psi \  | \  \bot \  |
   \  \Box_X \phi \  | \  \boxminus_X \phi
   \  \BP_X \phi \  | \  U \phi
 \  | \ 
   \circlearrowleft \]
It is important to note that our other fragments may be similarly extended.

Universal modality has the following semantics for Kripke structures: 
\[ \mathbb{M}, w \Vdash U \phi \iff \mathbb{M}, v \Vdash \phi\textup{
  for all $v \in W$} \]
Likewise, it has corresponding semantics for \textsc{EviL}
models:
\[ \Omega, w \VDash U \phi \iff \Omega, v \VDash \phi\textup{
  for all $v \in \Omega$} \]

As noted in \cite[pg. 67, Chapter 6]{van_benthem_modal_2010},
universal modality is closely related to the modal logic $S5$. Below,
we state the axioms for the univesal modality in \textsc{EviL}, which
are recognizably the axioms for $S5$, along with other axioms asserting
that the other relations are subrelations of $U$.

\begin{table}
\centering
%\newcounter{rownum}
\setcounter{rownum}{0}
%\newcounter{rownum2}
\setcounter{rownum2}{0}
\begin{tabular}{|ll|}
\hline
   (\refstepcounter{rownum}U\arabic{rownum})&$ \vdash U \phi \rightarrow  \phi$\\
  (\refstepcounter{rownum}U\arabic{rownum})&$ \vdash U \phi \to U U \phi$\\
  (\refstepcounter{rownum}U\arabic{rownum})&$ \vdash \neg U \phi \to U
  \neg U \phi$\\
  (\refstepcounter{rownum}U\arabic{rownum})\label{BoxU}&$ \vdash U
  \phi \to \Box_X \phi$\\
  (\refstepcounter{rownum}U\arabic{rownum})\label{BMU}&$ \vdash U
  \phi \to \BM_X \phi$\\
  (\refstepcounter{rownum}U\arabic{rownum})\label{BPU}&$ \vdash U
  \phi \to \BP_X \phi$\\
\hline
\end{tabular}
\caption{Axioms for Universal Modality}
\label{table:axiomsII}
\end{table}

We can think of the universal modality axioms (appropriately
restricted) as extending any of the
three systems we have looked at so far; \textsc{EviL} extends to
U\textsc{EviL},
\textsc{EviL}$^\BM$ extends to U\textsc{EviL}$^\BM$, and
\textsc{EviL}$^\BP$ extends to U\textsc{EviL}$^\BP$.  Abstract completeness
for all three systems is achieved in a similar manner.

\begin{theorem}[Universal \textsc{EviL} Completeness]\ \\
\begin{align*}
\Gamma \vdash_{U\textup{\textsc{EviL}}} \phi & \iff \Gamma\Vdash_{\textup{\textsc{EviL}}}
\phi \\
\Gamma \vdash_{U\textup{\textsc{EviL}$^\BM$}} \phi & \iff \Gamma\Vdash_{\textup{\textsc{EviL}}}
\phi \\
\Gamma \vdash_{U\textup{\textsc{EviL}}^\BP} \phi & \iff \Gamma\Vdash_{\textup{\textsc{EviL}}}
\phi 
\end{align*}
\end{theorem}
\begin{proof}
Soundness in all cases is straightforward.

For completeness, in each case we carry out the canonical model
construction, 
which will enforce that the accessibility relation associated with $U$ forms
a partition on the canonical model, and at every point within a
given partition, the other relations are a subrelation of the
candidate universal relation.  In each case the canonical model construction will
provide a world $w$ which witnesses $\Gamma$ but does not witness
$\phi$; to complete the construction, one need only take a \emph{point
  generated submodel} around $w$ \cite[chapter 2]{blackburn_modal_2001}, which preserves the truth of all
formulae at $w$ but establishes $U$ a universal modality.

From there, it is straightforward to verify that all of the bisimilar
model completions we have investigated preserve universal modality.
In each case, these may be used just as before to establish the
abstract completeness theorem desired.
\end{proof}

Just as the above proof illustrates our previous abstract completeness
theorems may be adapted, our finitary approaches may be modified to
accommodate universal modality as well.

\begin{theorem}[Small Model Property for Universal \textsc{EviL}]\ \\
For any universal \textsc{EviL} formula $\phi$, if it satifiable then
it is satisfiable in an \textsc{EviL} model with $O(EXP2(|\phi|))$
many worlds.
\end{theorem}
\begin{proof}
To see this, simply adapt the finite canonical model construction
$\invis$ we
previously saw in \ref{small-model}.  An important modification to
this construction is to define:
\[ w U v \iff (U \phi \in w \iff U \phi \in v)\]

This will enforce that $U$ forms an $S5$ modality; see \cite[chapter
5, pgs. 81--82]{boolos_logic_1995} for details.  

To ensure that the other relations are subrelations of $U$, 
one needs to ensure that $\Sigma$ is defined as follows:
\begin{align*}
\Sigma(\Delta,U \phi) := &   \{ U \phi, \neg U \phi \} \cup \\
& \{ \Nec_X \phi,\neg \Nec_X \phi, \\
& \ \BB_X \phi, \neg\BB_X \phi, \\
& \ \BBI_X \phi, \neg\BBI_X \phi  \ |\ X \in \Delta
\}
\end{align*}
Given $\Sigma$ constructed in this fashion, one may readily verify
that the Universal
\textsc{EviL} axioms enforce the other relations  
\end{proof}
%%% Local Variables: 
%%% mode: latex
%%% TeX-master: "evil_philosophy"
%%% End: 
