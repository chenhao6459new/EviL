In this section, we take stock of what we have illustrated so far in
our investigations into the completeness of \textsc{EviL}.  We discuss
how the nature of the abstract semantics for \textsc{EviL} in
relationship to its concrete semantics, and we view this relationship
from a wider mathematical perspective.

We shall begin by substantiating the relationship we established in
\S\ref{taking-stock}, which we expressed in \eqref{relationship}.

\begin{lemma} For finite $\Gamma$:
\[
\Gamma \Vdash_{\textsc{EviL}} \phi \iff \Gamma \VDash \phi
\]
\end{lemma}
\begin{proof}
We may observe that since $\Gamma$ is finite, then by classical logic
and our previous completeness theorems we have the following chain of reasoning:
\begin{align*}
  \Gamma \Vdash_{\textsc{EviL}} \phi & \iff  \Vdash_{\textsc{EviL}} \bigwedge \Gamma
  \to \phi \\
   & \iff \vdash_{\textsc{EviL}} \bigwedge \Gamma
  \to \phi \\
   & \iff \VDash \bigwedge \Gamma
  \to \phi \\
   & \iff  \Gamma \VDash \phi
\end{align*}
\end{proof}

As a further remark, we feel the need to discuss the nature of the
relationship between the \emph{concrete} and \emph{abstract} semantics
that \textsc{EviL} exhibits.  We began with \textsc{EviL} models,
which were intended to model intuitions we had regarding the nature of
epistemology.  In so doing, we used the language of traditional 
epistemic logic, even though we modified the semantics heavily.  We
found this gave rise to relational models that are the traditional
object of study of modal logic, however we found that while we could
abstract to traditional Kripke structures, this was not symmetric --
we could not abstract back.

We argue that this particular relationship is common place in
mathematics.  For instance, it is natural to think of the integers as
a concrete object.  After all, every mathematics student at some point
learns Kronicker's legendary quote ``God created the integers, all
else is the creation of man'' \cite[pg. 477]{bell_men_1986}.
However, it is by these concrete origins, we may recognize the
integers as concrete Noetherian ring.  Indeed, it is by understanding
the integers that the theory of Noetherian rings proceeds.  For
instance, the fact that every ideal in a Noetherian ring is equal to a finite
intersection of primary ideals is a pure abstraction of 
Euclid's prime decomposition theorem\cite[Lemmas 7.11 and 7.12,
pg. 83]{atiyah_introduction_1994}.  This is part of the character of
mathematics; abstraction is guided by intuition drawn from more
concrete objects.  In  the same manner we may regard Stone
Representation Theorem as abstracting
Birkoff's theorem \cite[chapters 11 and 5, respectively]{davey_introduction_2002}, and the Yoneda Lemma abstracting Cayley's
theorem \cite[chapters 4 and 1, respectively]{smith_post-modern_1999}.

Despite the order of presentation given here, we should make things
clear - we did not derive the abstract completeness theorem in
\S\ref{abstraction} until we were convinced that \textsc{EviL} Kripke
structures generalized our concrete structures.  The process by which
\textsc{EviL} was developed involved finding the results in
\S\ref{small-model} first, and letting those properties we deemed 
necessary to coerce a Kripke structure into a \textsc{Evil} 
model define the logic.  
Abstract completeness was an afterthought.  Of course, just as in the
case of complex analysis and trigonometry, our abstract formalism
is far easier to manipulate than our original \textsc{EviL} models.

Since \textsc{EviL} Kripke structures really are abstract
idealizations of concrete \textsc{EviL} models, as we have
illustrated, we are granted a particularly comfortable situation.  
On the one hand, we have concrete semantics by
which we may sharpen our intuition. On the other hand, we 
have well behaved abstract semantics which faithfully provide an 
idealized domain for us to carry out formal work with relative ease.
%%% Local Variables: 
%%% mode: latex
%%% TeX-master: "evil_philosophy"
%%% End: 
