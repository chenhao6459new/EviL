In the previous sections, we showed that the logic of \textsc{EviL} we
presented in \ref{evil-axioms} was complete for \textsc{EviL} Kripke
models.  In this section we pause for a moment to discuss why we must
go further, and reason what further needs to be shown in order to
establish completeness.

First, recall the semantics we developed in
Definition \ref{evil-semantics-def} in \S\ref{evil-grammar}.
These semantics were carefully crafted to make true the mystical
Theorem \ref{theorem-theorem}, the \emph{Theorem Theorem}.  This
equated $\Box \phi$ with a proof of $\phi$, in the following manner:

\[ \mathfrak{M},(a,A) \VDash \Box_X \phi \iff Th(\mathfrak{M}) \cup A_X \vdash_{\textsc{EviL}} \phi \]

In the above, we assume that $A_X$ is finite. This above property of \textsc{EviL} was enforced to accommodate the \emph{Justification
  Principle} from \S\ref{explicit}, which says that when an agent
believes something, she must have a reason.

This critical insight, driven by our philosophical perspective on
the nature of knowledge, is lost in the abstracta of Kripke
semantics.  The Kripke semantics perspective on
\textsc{EviL} is basically meaningless on its own; 
for why would anyone ever care about \textsc{EviL} Kripke structures,
without the light of the fact that they somehow 
abstract \textsc{EviL} semantics?  
We know that not every \textsc{EviL} Kripke structure 
can be represented by a \textsc{EviL} structure by Proposition
\ref{not-an-evil-model}.  How do we know that \textsc{EviL} Kripke
structures are faithfully abstracting our concrete semantics at all?

The connection of \textsc{EviL} Kripke structures to \textsc{EviL} has
not yet been entirely revealed, but it is as follows: 
\begin{quote}
\emph{\textsc{EviL} models are finitary, concrete objects, and
  \textsc{EviL} Kripke structures are their potentially infinite, abstract idealizations.}
\end{quote}
Succinctly, this relationship is expressed as follows:
\begin{equation} 
\Gamma \Vdash_{\textsc{EviL}} \phi \iff \Gamma \VDash \phi \label{relationship}
\end{equation}
\ldots for all $\phi$ and finite $\Gamma$.

The relationship is important, since Kripke Semantics are the natural
semantics for modal logics, and hence enable one to rapidly reason
about them.  Equation \eqref{relationship} allows us to see  
that, when thinking about \textsc{EviL}, one 
may freely employ strong completeness and neglect
concerns about failure of compactness, with the understanding that
when we restrict ourselves to  finitary circumstances the abstract
semantics and the concrete semantics coincide.

Sections \S\ref{small-model} through \S\ref{translation} shall be
devoted to establishing this relationship between the abstract and
concrete semantics for \textsc{EviL}.
Since we know that the logic of \textsc{EviL} models is not compact
from Theorem \ref{noncompact}, we shall establish a
\emph{small model property} for partly \textsc{EviL} and \textsc{EviL}
Kripke structures in \S\ref{small-model}.  
By modifying the translation system for finite Kripke structures to
\textsc{EviL} modifying we originally gave in the proof of
Proposition \ref{translation-sketch} from \S\ref{sketch}, we shall
show how to translate finite \textsc{EviL} Kripke structures into
\textsc{EviL} models in \S\ref{translation}.  We shall find need to
make use of the concept of \emph{islands}, which we shall introduce in \S\ref{islands}.

After the above developments, we shall once again take stock of our
observations in \S\ref{taking-stockII}.  We shall prove the equation
\eqref{relationship}, and make use of our previous results to
establish complexity bounds on the decision procedure for \textsc{EviL}.

%%% Local Variables: 
%%% mode: latex
%%% TeX-master: "evil_philosophy"
%%% End: 
