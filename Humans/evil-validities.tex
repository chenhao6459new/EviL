The previous philosophical readings of \textsc{EviL} immediately
suggest certain validities will hold in the semantics.  For instance, the
assertion ``A set of premises is sound if and only if all of its
subsets are sound.'' would be expressed as
\begin{equation}
\VDash \PP \IFF \BM \PP \label{ppequiv}
\end{equation}
Indeed, this is a validity of \textsc{EviL}.  Schematically, it may be
tempting to think that maybe the same is true for $\BP$.  However, we
have that:
\begin{equation}
\nVDash \PP \to \BP \PP \label{notPPBPPP}
\end{equation}
Nor does this make much sense.  It asserts ``If the agent's basic beliefs
are sound, then all extensions of her basic beliefs are sound too.''
Soundness is a fragile thing -- it is rather easy to think of things
to add to a sound set of basic beliefs which break soundness, such as
``All logicians are centaurs''  and other such demonstrably false nonsense.

Related to \eqref{ppequiv}, there is another related validity  
associated with $\PP$; namely that if the
\textsc{EviL} agent's assumptions are sound, then anything she
concludes from them is true (employing the reading which naturally
arises from Theorem \ref{theorem-theorem}).  This is expressed as
\begin{equation}
\VDash \PP \to \Box \phi \to \phi \label{axiom-11}
\end{equation}
The formula \eqref{ppequiv} expresses that the soundness of one's
premises  is something \emph{persistent} as the \textsc{EviL} agent
carries on casting doubt on assumptions and discarding them.  Another
thing that is persistent this way is the \text{EviL} agent's
imagination:
\begin{equation}
\VDash \Pos \phi \to \BM \Pos \phi \label{axiom-8}
\end{equation}
One may read \eqref{axiom-8} as saying something like ``If the \textsc{EviL}
agent can imagine/conceive of something, then no matter what things she casts into
doubt, she can still imagine it.''  One can also express something
like the dual of this, namely
\begin{equation}
\VDash \Box \phi \to \BP \Box \phi
\end{equation}
We shall read the above as asserting ``If the agent \emph{can compose
  an argument} then she will still be able to compose that argument if
she remembers more information and experiences she's had in the world.''  
This should not be surprising -- this is yet another expression of the 
Theorem \ref{theorem-theorem}, the Theorem Theorem, and the fact that
the proof theory of \textsc{EviL} is monotonic.
In general, many of the assertions
here exhibit interplay between $\BP$ and $\Box$, and dually $\BM$ and
$\Pos$  -- further investigation of these relationships is taken up in \S\ref{elimination}.

For better or for worse, \textsc{EviL} semantics make
true the following assertion: if something is achievable by repeatedly casting
assumptions into doubt, then it's achievable by casting assumptions
into doubt only once:
\begin{equation}
\VDash \PM^+\phi \to \PM \phi
\end{equation}
Here $^+$ is taken from the syntax for \emph{regular
  expressions} commonly used in computer science and UNIX programming
to mean ``one or
more'' \citep{friedl_mastering_2006}.  Similarly, we have assumed that
discarding no assumptions is, in a way, vacuously casting assumptions
into doubt.  In light of this \textsc{EviL} also makes true the following:
\begin{equation}
\VDash \phi \to \PM \phi
\end{equation}
Furthermore, it is worth mentioning some harder to understand
validities of this system.  The first one is that when the agent
believes something, they believe it regardless of the process of
doubting or embracing their beliefs:
\begin{eqnarray}
\VDash \Box \phi \to \Box \BM \phi \label{dontcare1}\\
\VDash \Box \phi \to \Box \BP \phi \label{dontcare2}
\end{eqnarray}
We can observe that this generalizes to multiple agents, as specified
in \S\ref{multi-agent}.

Another more challenging validity is the fact that if
some proposition $\phi$ holds, then for any restriction of the
\textsc{EviL} agent's beliefs (or dually, any extension), 
if those beliefs are sound, then $\phi$ must be conceivable (i.e.,
$\Pos \phi$ holds).  This is
expressed as the following two validities:
\begin{eqnarray}
\VDash \phi \to \BM (\PP \to \Pos \phi) \label{fraghack1}\\
\VDash \phi \to \BP (\PP \to \Pos \phi) \label{fraghack2}
\end{eqnarray}
Finally, another a peculiarity of \textsc{EviL} is that not all of its
validities are \emph{schematic}.  For instance, there is a kind of
\emph{Cartesian dualism} present in the semantics, where the
\textsc{EviL} agent's deliberation on her evidence does not bear on
brute matters of fact.  For a world pair $(a,A)$, $A$ and $a$ are
basically separate - an \textsc{EviL} agent's mind and the world they
live are composed of different substance.  This gives rise to the following four validities: 
\begin{eqnarray}
& \VDash p \to \BM p \\
& \VDash p \to \BP p \\
& \VDash \neg p \to \BM \neg p \\
& \VDash \neg p \to \BP \neg p
\end{eqnarray}
Hence, \textsc{EviL} is not a \emph{normal} logic. 
%  This should
% admittedly be considered a wart on the semantics, since it appears
% that it rules out the conventional algebraic duality most modal logics
% exhibit (see \citet{blackburn_modal_2001}, chapter 5).  

On the other hand, it is by the same assumption of Cartesian dualism
that underlies the non-normality that \eqref{ppequiv}
as is a natural consequence.  By accepting non-normality, and the
grammar restriction we have imposed on \emph{basic beliefs} to avoid
paradoxes, it follows as a consequence that a belief set is sound if
and only if all of its subsets are sound.  Hence non-normality for
\textsc{EviL} part of the price that must be payed % -- it compromises the algebraic elegance
%of the semantics, while simultaneously giving rise
for the philosophically appealing features that the logic has to offer.

In the next section, we turn to a more systematic study of the
validities of \textsc{EviL}.  We shall see that this gives rise to an
\emph{elimination theorem}.
%%% Local Variables: 
%%% mode: latex
%%% TeX-master: "evil_philosophy"
%%% End: 
