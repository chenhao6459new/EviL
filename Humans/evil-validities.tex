The previous philosophical readings of \textsc{EviL} immediately
suggest certain validities will hold the semantics.  For instance, the
assertion ``A set of premises is sound if and only if all of its
subsets are sound.'' would be expressed as
\begin{equation}
\VDash \PP \IFF \BM \PP \label{ppequiv}
\end{equation}
\ldots and indeed, this is a validity of \textsc{EviL}.  There is
another, related validity associated with $\PP$; namely that if the
\textsc{EviL} agent's assumptions are sound, then anything she
concludes from them is true (employing the reading which naturally
arises from Theorem \ref{theorem-theorem}).  This is expressed as
\begin{equation}
\VDash \PP \to \Box \phi \to \phi \label{axiom-11}
\end{equation}
The formula \eqref{ppequiv} expresses that the soundness of one's
premises  is something \emph{persistent} as the \textsc{EviL} agent
carries on casting doubt on assumptions and discarding them.  Another
thing that is persistent this way is the \text{EviL} agent's
imagination:
\begin{equation}
\VDash \Pos \phi \to \BM \Pos \phi \label{axiom-8}
\end{equation}
I read \eqref{axiom-8} as saying something like ``If the \textsc{EviL}
agent can imagine something, then no matter things she casts into
doubt, she can still imagine it.''  One can also express something
like the dual of this, namely
\begin{equation}
\VDash \Box \phi \to \BP \Box \phi
\end{equation}
\ldots which I read as asserting ``If th agent can compose an argument
then she'll still be able to compose that argument if she remembers
more premises she has available.''  In general, many of the assertions
here have an interplay like this -- interest in these relationships is
taken up in \S\ref{elimination}.

Furthermore, for better or for worse the \textsc{EviL} semantics make
true the following: if something is achievable by repeatedly casting
assumptions into doubt, then it's achievable by casting assumptions
into doubt only once:
\begin{equation}
\VDash \PM^+\phi \to \PM \phi
\end{equation}
\ldots where $^+$ is taken from the syntax for \emph{regular
  expressions} commonly used in computer science and UNIX programming
to mean ``one or
more'' \citep{friedl_mastering_2006}.  Similarly, I have assumed that
discarding no assumptions is, in a way, vacuously casting assumptions
into doubt.  In light of this \textsc{EviL} also makes true the following:
\begin{equation}
\VDash \phi \to \PM \phi
\end{equation}
Furthermore, it is worth mentioning some harder to understand
validities of this system.  The first one is that when the agent
believes something, they believe it regardless of the process of
doubting or embracing their beliefs:
\begin{eqnarray}
\VDash \Box \phi \to \Box \BM \phi \label{dontcare1}\\
\VDash \Box \phi \to \Box \BP \phi \label{dontcare2}
\end{eqnarray}
We can observe that this generalizes to multiple agents, as specified
in \S\ref{multi-agent}.

Another more challenging validity is the fact that if
some proposition $\phi$ holds , then for any restriction of
\textsc{EviL} agent's beliefs (or dually, any extension), 
if those beliefs are sound, then $\phi$ must be conceivable.  This is
expressed as the following two validities:
\begin{eqnarray}
\VDash \phi \to \BM (\PP \to \Pos \phi) \label{fraghack1}\\
\VDash \phi \to \BP (\PP \to \Pos \phi) \label{fraghack2}
\end{eqnarray}
Finally, another a peculiarity of \textsc{EviL} is that not all of its
validities are \emph{schematic}.  For instance, there is a kind of
\emph{Cartesian dualism} present in the semantics, where the
\textsc{EviL} agent's deliberation on her evidence does not bear on
brute matters of fact.  For a world pair $(a,A)$, $A$ and $a$ are
basically separate - an \textsc{EviL} agent's mind and the world they
live are composed of different substance.  This gives rise to the following four validities: 
\begin{eqnarray}
& \VDash p \to \BM p \\
& \VDash p \to \BP p \\
& \VDash \neg p \to \BM \neg p \\
& \VDash \neg p \to \BP \neg p
\end{eqnarray}
Hence, \textsc{EviL} is not a \emph{normal} logic.  On the other hand,
the duality between doubting and embracing one's experience, belief
and imagination, as well as the soundness and unsoundness give rise
to an interplay which I don't believe is present in any \emph{normal} logic.
This shall be the subject of study in \S\ref{elimination}.
%%% Local Variables: 
%%% mode: latex
%%% TeX-master: "evil_philosophy"
%%% End: 
