% \documentclass{letter}
% \usepackage{geometry,amsmath,amssymb,bbm}
% \geometry{letterpaper}
% 
% %%%%%%%%%% Start TeXmacs macros
% \newcommand{\tmop}[1]{\ensuremath{\operatorname{#1}}}
% \newcommand{\tmtextit}[1]{{\itshape{#1}}}
% \newcommand{\tmtextsc}[1]{{\scshape{#1}}}
% \newenvironment{itemizedot}{\begin{itemize} \renewcommand{\labelitemi}{$\bullet$}\renewcommand{\labelitemii}{$\bullet$}\renewcommand{\labelitemiii}{$\bullet$}\renewcommand{\labelitemiv}{$\bullet$}}{\end{itemize}}
% %%%%%%%%%% End TeXmacs macros
% 
% \begin{document}

These properties are definitive - as we shall demonstrate, \tmtextsc{EviL} is
sound and weakly complete for \tmtextsc{EviL} models.

The Kripke semantics also serve to \ provide proper intuition behind
\tmtextsc{EviL} models. \ I think of the defined relations given as follows:
\begin{itemizedot}
  \item If $x R^{\Omega}_X y$, then at world $x$ the agent $X$ can imagine $y$
  is true, since $y$ is compatible with what the agent believes
  
  \item If $x \sqsubseteq^{\Omega}_X y$, then at world $x$, agent $X$'s
  assumptions (or the experiences they are taking under consideration) are
  contained in her evidence at $y$
\end{itemizedot}
Given this perspective, the proof of \ref{pV} can be understood in the
following way - if the agent assumes fewer things, more things are imaginable,
since it's easier for a world to be incompatible with an agent's evidence.

In fact, in light of Theorem \ref{theorem-theorem}, the above follows
from a more general relationship present in model theory.  For a given
Kripke structure $\mathbbm{M}$, define two operators $Mod^{\mathbbm{M}}
: \powerset  \mathcal{L} (\Phi, \mathcal{A}) \rightarrow \powerset
(W^{\mathbbm{M}})$ and $Th^{\mathbbm{M}} : \powerset
(W^{\mathbbm{M}}) \rightarrow \powerset \mathcal{L} (\Phi, \mathcal{A})$
\begin{align*}
  Mod^{{\mathbbm{M}}}({\Delta}) &
  =\{x{\in}W{\ }|{\ }{\forall}{\psi}{\in}{\Delta}.
  {\mathbbm{M}},x{\Vdash}{\psi}\}\\
  Th^{{\mathbbm{M}}}({\nabla}) & =\{{\psi}{\in}\mathcal{L}({\Phi},
  \mathcal{A}){\ }|{\ }{\forall}x{\in}{\nabla}.
  {\mathbbm{M}},x{\Vdash}{\psi}\}
\end{align*}
We then have, for any $\Delta \in \powerset  \mathcal{L}
(\Phi, \mathcal{A})$ and $\nabla \in \powerset (W^{\mathbbm{M}})$:
\[ \nabla \subseteq Mod^{\mathbbm{M}} (\Delta) \text{ if and only if }
   \Delta \subseteq Th^{\mathbbm{M}} (\nabla) \]
$\ldots$hence we these two operations form what is refered an
\tmtextit{antitone Galois connection}, between the lattice $\powerset
(W^{\mathbbm{M}})$ and the lattice $\powerset \mathcal{L} (\Phi,
\mathcal{A})$. It follows from the theory of Galois connections
\citep[][chapter 3]{roman_lattices_2008} that
the following two properties:
\begin{itemizedot}
  \item If $\nabla \supseteq \nabla'$ then $Th^{\mathbbm{M}} (\nabla)
  \subseteq Th^{\mathbbm{M}} (\nabla')$
  \item If $\Delta \supseteq \Delta'$ then $Mod^{\mathbbm{M}} (\Delta)
  \subseteq Mod^{\mathbbm{M}} (\Delta')$
\end{itemizedot}
We can see that (\ref{pV}) follows from the second of these two properties. \
To see this, assume that $(a, A) \sqsupseteq_X^{\Omega} (b, B)$, hence $A_X
\supseteq B_X$ and thus $Mod^{\Omega} (A_X) \subseteq
Mod^{\Omega} (B_X)$. \ But it follows from semantics of \tmtextsc{EviL}
we have that $(c, C) \in Mod^{\Omega} (A_X)$ if and only if $(a, A)
R^{\Omega}_X (c, C)$, and likewise for $B_X$. \ Hence if $(a, A) R^{\Omega}_X
(c, C)$ then $(b, B) R^{\Omega}_X (c, C)$.

% \end{document}