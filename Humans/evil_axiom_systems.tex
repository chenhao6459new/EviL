In this section, we shall present the axiom system which represents
the validities of \textsc{EviL} semantics as provided in . 

Table \ref{table:axioms}, provides a Hilbert-style axiom system for \textsc{EviL}.  In addition to giving each axiom, we have also provided our own philosophical reading of what each axiom says.  One unusual feature of this logic is that it is not \emph{normal}, that is it is not closed under variable substitution.

\begin{table}
\newcounter{rownum}
\setcounter{rownum}{0}
\newcounter{rownum2}
\setcounter{rownum2}{0}
\begin{tabularx}{\linewidth}{|cl>{\it}X|}
\hline
(\addtocounter{rownum}{1}\arabic{rownum}) & $\vdash \phi \to \psi \to \phi$ & \multirow{3}{*}{Axioms for basic propositional logic} \\
(\addtocounter{rownum}{1}\arabic{rownum}) & $\vdash (\phi \to \psi \to \chi) \to (\phi \to \psi) \to \phi \to \chi$ &  \\
(\addtocounter{rownum}{1}\arabic{rownum}) & $\vdash (\neg \phi \to \neg \psi) \to \psi \to \phi$ &  \\[6pt]
(\addtocounter{rownum}{1}\arabic{rownum}) & $\vdash \BBI_X \phi \to \phi$ & If $\phi$ holds under any further evidence $X$ considers, then $\phi$ holds simpliciter, since considering no additional evidence is trivially considering further evidence \\[6pt]
(\addtocounter{rownum}{1}\arabic{rownum}) & $\vdash \BBI_X \phi \to \BBI_X \BBI_X \phi$ & If $\phi$ holds under any further evidence $X$ considers, then $\phi$ holds whenever $X$ considers even further evidence beyond that \\[6pt]
(\addtocounter{rownum}{1}\arabic{rownum}) & $\vdash p \to \BB_X p$ & \multirow{2}{8.5cm}{Changing one's mind does not bear on matters of fact}\\
(\addtocounter{rownum}{1}\arabic{rownum}) & $\vdash p \to \BBI_X p$ & \\[6pt]
(\addtocounter{rownum}{1}\arabic{rownum}) & $\vdash \Pos_X \phi \to \BB_X \Pos_X \phi$ & The more evidence $X$ discards, the freer her imagination can run \\[6pt]
(\addtocounter{rownum}{1}\arabic{rownum}) & $\vdash \Nec_X \phi \to \Nec_X \BB_Y \phi$ &  \multirow{2}{8.5cm}{If $X$ believes a proposition, she believes it regardless of what anyone else thinks} \\
(\addtocounter{rownum}{1}\arabic{rownum}) & $\vdash \Nec_X \phi \to \Nec_X \BBI_Y \phi$ & \\[6pt]
(\addtocounter{rownum}{1}\arabic{rownum}) & $\vdash \PP_X \to \Nec_X \phi \to \phi$ & If $X$'s premises are sound, then her logical conclusion are correct \\[6pt]
(\addtocounter{rownum}{1}\arabic{rownum}) & $\vdash \PP_X \to \BB_X \PP_X $ & If $X$'s premises are sound then any subset of will be sound as well \\[6pt]
(\addtocounter{rownum}{1}\arabic{rownum}) & $\vdash \phi \to \BB_X \DDI_X \phi$ &
\multirow{2}{8.5cm}{Embracing evidence is the inverse of discarding evidence} 
\\
(\addtocounter{rownum}{1}\arabic{rownum}) & $\vdash \phi \to \BBI_X \DD_X \phi$ & \\[6pt]
(\addtocounter{rownum}{1}\arabic{rownum}) & $\vdash \Nec_X (\phi \to \psi) \to \Nec_X \phi \to \Nec_X \psi$ &
\multirow{3}{8.5cm}{Variations on axiom $K$}
\\
(\addtocounter{rownum}{1}\arabic{rownum}) & $\vdash \BB_X (\phi \to \psi) \to \BB_X \phi \to \BB_X \psi$ & \\
(\addtocounter{rownum}{1}\arabic{rownum}) & $\vdash \BBI_X (\phi \to \psi) \to \BBI_X \phi \to \BBI_X \psi$ & \\[6pt]
(\addtocounter{rownum2}{1}\Roman{rownum2}) & 
 $\AxiomC{$\vdash \phi \to \psi$}
\AxiomC{$\vdash \phi$}
\BinaryInfC{$\vdash \psi$}
\DisplayProof$ & Modus Ponens\\[10pt]
(\addtocounter{rownum2}{1}\Roman{rownum2}) & 
 $\AxiomC{$\vdash \phi$}
\UnaryInfC{$\vdash \Box_X \phi$}
\DisplayProof$ & \multirow{3}{8.5cm}{Variations on necessitation}\\
(\addtocounter{rownum2}{1}\Roman{rownum2}) & 
 $\AxiomC{$\vdash \phi$}
\UnaryInfC{$\vdash \BB_X \phi$}
\DisplayProof$ &  \\
(\addtocounter{rownum2}{1}\Roman{rownum2}) &
 $\AxiomC{$\vdash \phi$}
\UnaryInfC{$\vdash \BBI_X \phi$}
\DisplayProof$ & \\[10pt]
\hline
\end{tabularx}
\caption{A Hilbert style axiom system for \textsc{EviL}}
\label{table:axioms}
\end{table}


%%% Local Variables: 
%%% mode: latex
%%% TeX-master: "../evil_philosophy"
%%% End: 


This logic makes true a variety of relationships between the various
modalities, which are given in the following lemma:
\begin{lemma}\label{equivs}
We have the following provable equivalences:
\begin{eqnarray*} \vdash \Nec_X \phi \IFF \BB_X \Nec_X \phi & \hspace{1cm} \vdash \Nec_X \phi \IFF \Nec_X \BB_Y \phi \hspace{1cm}  & \vdash \Nec_X \phi \IFF \Nec_X \BBI_Y \phi \\
 \vdash \BB_X \phi \IFF \BB_X \BB_X \phi & \vdash \BBI_X \phi \IFF \BBI_X \BBI_X \phi & \vdash \PP_X \IFF \BB_X \PP_X\end{eqnarray*}
%&\vdash \PP_X \IFF \DDI_X \PP_X \\
%&\vdash \phi \IFF \psi \textup{ implies } \vdash \chi \IFF \chi[\phi/\psi]
%\end{eqnarray*}
%\ldots where $\chi[\phi/\psi]$ is the same as $\chi$ but every occurrence of $\phi$ as a subformula is replaced by $\psi$.
%$\hfill\dashv$
\end{lemma}

In addition to the main system presented above, it can be understood to contain two subsystems, corresponding to two fragments of the main grammar:
\begin{definition}
Define $\mathcal{L}^\boxminus (\Phi, \mathcal{A})$ as the fragment:
\[ \phi \ {::=} \  p \in \Phi \  | \  \phi
   \rightarrow \psi \  | \  \bot \  |
   \  \Box_X \phi \  | \  \boxminus_X \phi
%   \  | \  \boxplus_X \phi
 \  | \ 
   \circlearrowleft_X \]

And define $\mathcal{L}^\boxplus (\Phi, \mathcal{A})$ as the fragment:
%Define $\mathcal{L}^\boxminus (\Phi, \mathcal{A})$ as the fragment:
\[ \phi \ {::=} \  p \in \Phi \  | \  \phi
   \rightarrow \psi \  | \  \bot \  |
   \  \Box_X \phi 
%\  | \  \boxminus_X \phi
   \  | \  \boxplus_X \phi
 \  | \ 
   \circlearrowleft_X \]
\end{definition}

Table \ref{table:axiomsII} gives the axioms systems for these two fragments.  For now, we shall observe that \textsc{EviL} extends \textsc{EviL}$^\BM$ and \textsc{EviL}$^\BP$.  In \S\ref{conservative-extension} we shall make this precise. 
\begin{table}
\begin{minipage}[b]{0.5\linewidth}
\centering
%\newcounter{rownum}
\setcounter{rownum}{0}
%\newcounter{rownum2}
\setcounter{rownum2}{0}
\begin{tabular}{|ll|}
\hline
  (\addtocounter{rownum}{1}\arabic{rownum})&$ \vdash \phi \rightarrow \psi \rightarrow \phi$\\
  (\addtocounter{rownum}{1}\arabic{rownum})&$ \vdash (\phi \rightarrow \psi \rightarrow \chi) \rightarrow (\phi
  \rightarrow \psi) \rightarrow \phi \rightarrow \chi$\\
  (\addtocounter{rownum}{1}\arabic{rownum})&$ \vdash (\neg \phi \rightarrow \neg \psi) \rightarrow \psi \rightarrow
  \phi$\\
  (\addtocounter{rownum}{1}\arabic{rownum})&$ \vdash \boxminus_X \phi \rightarrow \phi$\\
  (\addtocounter{rownum}{1}\arabic{rownum})&$ \vdash \boxminus_X \phi \rightarrow \boxminus_X \boxminus_X \phi$\\
  (\addtocounter{rownum}{1}\arabic{rownum})&$ \vdash p \rightarrow \boxminus_X p$\\
  (\addtocounter{rownum}{1}\arabic{rownum})&$ \vdash \neg p \rightarrow \boxminus_X \neg p$\\
  (\addtocounter{rownum}{1}\arabic{rownum})&$ \vdash \diamondsuit_X \phi \rightarrow \boxminus_X \diamondsuit_X \phi$\\
  (\addtocounter{rownum}{1}\arabic{rownum})&$ \vdash \Box_X \phi \rightarrow \Box_X \boxminus_Y \phi$\\
  (\addtocounter{rownum}{1}\arabic{rownum})&$ \vdash \phi \rightarrow \boxminus_X (\circlearrowleft_X \rightarrow
  \diamondsuit_X \phi)$\\
  (\addtocounter{rownum}{1}\arabic{rownum})&$ \vdash \circlearrowleft_X \rightarrow \boxminus_X \circlearrowleft_X$\\
  (\addtocounter{rownum}{1}\arabic{rownum})&$ \vdash \Box_X (\phi \rightarrow \psi) \rightarrow \Box_X \phi \rightarrow
  \Box_X \psi$\\
  (\addtocounter{rownum}{1}\arabic{rownum})&$ \vdash \boxminus_X (\phi \rightarrow \psi) \rightarrow \boxminus_X \phi
  \rightarrow \boxminus_X \psi$\\
(\addtocounter{rownum2}{1}\Roman{rownum2}) & 
 $\AxiomC{$\vdash \phi \to \psi$}
\AxiomC{$\vdash \phi$}
\BinaryInfC{$\vdash \psi$}
\DisplayProof$ \\ %& Modus Ponens\\[10pt]
(\addtocounter{rownum2}{1}\Roman{rownum2}) & 
 $\AxiomC{$\vdash \phi$}
\UnaryInfC{$\vdash \Box_X \phi$}
\DisplayProof$ \\ %& \multirow{3}{8.5cm}{Variations on necessitation}\\
(\addtocounter{rownum2}{1}\Roman{rownum2}) & 
 $\AxiomC{$\vdash \phi$}
\UnaryInfC{$\vdash \BB_X \phi$}
\DisplayProof$   \\
% (\addtocounter{rownum2}{1}\Roman{rownum2}) &
%  $\AxiomC{$\vdash \phi$}
% \UnaryInfC{$\vdash \BBI_X \phi$}
% \DisplayProof$  \\% [10pt]
\hline
\end{tabular}
\end{minipage}
\hspace{0.5cm}
\begin{minipage}[b]{0.5\linewidth}
 \centering
%\newcounter{rownum}
\setcounter{rownum}{0}
%\newcounter{rownum2}
\setcounter{rownum2}{0}
\begin{tabular}{|ll|}
\hline
  (\addtocounter{rownum}{1}\arabic{rownum})&$ \vdash \phi \rightarrow \psi \rightarrow \phi$\\
  (\addtocounter{rownum}{1}\arabic{rownum})&$ \vdash (\phi \rightarrow \psi \rightarrow \chi) \rightarrow (\phi
  \rightarrow \psi) \rightarrow \phi \rightarrow \chi$\\
  (\addtocounter{rownum}{1}\arabic{rownum})&$ \vdash (\neg \phi \rightarrow \neg \psi) \rightarrow \psi \rightarrow
  \phi$\\
  (\addtocounter{rownum}{1}\arabic{rownum})&$ \vdash \boxplus_X \phi \rightarrow \phi$\\
  (\addtocounter{rownum}{1}\arabic{rownum})&$ \vdash \boxplus_X \phi \rightarrow \boxplus_X \boxplus_X \phi$\\
  (\addtocounter{rownum}{1}\arabic{rownum})&$ \vdash p \rightarrow \boxplus_X p$\\
  (\addtocounter{rownum}{1}\arabic{rownum})&$ \vdash \neg p \rightarrow \boxplus_X \neg p$\\
  (\addtocounter{rownum}{1}\arabic{rownum})&$ \vdash \Box_X \phi \rightarrow \boxplus_X \Box_X \phi$\\
  (\addtocounter{rownum}{1}\arabic{rownum})&$ \vdash \Box_X \phi \rightarrow \Box_X \boxplus_Y \phi$\\
  (\addtocounter{rownum}{1}\arabic{rownum})&$ \vdash \phi \rightarrow \boxplus_X (\circlearrowleft_X \rightarrow
  \diamondsuit_X \phi)$\\
  (\addtocounter{rownum}{1}\arabic{rownum})&$ \vdash \neg \circlearrowleft_X \rightarrow \boxplus_X \neg
  \circlearrowleft_X$\\
  (\addtocounter{rownum}{1}\arabic{rownum})&$ \vdash \Box_X (\phi \rightarrow \psi) \rightarrow \Box_X \phi \rightarrow
  \Box_X \psi$\\
  (\addtocounter{rownum}{1}\arabic{rownum})&$ \vdash \boxplus_X (\phi \rightarrow \psi) \rightarrow \boxplus_X \phi
  \rightarrow \boxplus_X \psi$\\
(\addtocounter{rownum2}{1}\Roman{rownum2}) & 
 $\AxiomC{$\vdash \phi \to \psi$}
\AxiomC{$\vdash \phi$}
\BinaryInfC{$\vdash \psi$}
\DisplayProof$ \\ %& Modus Ponens\\[10pt]
(\addtocounter{rownum2}{1}\Roman{rownum2}) & 
 $\AxiomC{$\vdash \phi$}
\UnaryInfC{$\vdash \Box_X \phi$}
\DisplayProof$ \\ %& \multirow{3}{8.5cm}{Variations on necessitation}\\
% (\addtocounter{rownum2}{1}\Roman{rownum2}) & 
%  $\AxiomC{$\vdash \phi$}
% \UnaryInfC{$\vdash \BB_X \phi$}
% \DisplayProof$   \\
(\addtocounter{rownum2}{1}\Roman{rownum2}) &
 $\AxiomC{$\vdash \phi$}
\UnaryInfC{$\vdash \BBI_X \phi$}
\DisplayProof$  \\% [10pt]
\hline
\end{tabular}
\end{minipage}
\caption{Axiom system \textsc{EviL}$^\BM$ and \textsc{EviL}$^\BP$ respectively}
\label{table:axiomsII}
\end{table}

%%% Local Variables: 
%%% mode: latex
%%% TeX-master: "evil_philosophy"
%%% End: 
