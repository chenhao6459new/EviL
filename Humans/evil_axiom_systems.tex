In this section, we shall present the axiomatics for multi-agent
\textsc{EviL}. The axioms of \textsc{EviL} are presented in 
Table \ref{table:axioms}, which comprises a Hilbert systems for
$\vdash_{\textsc{EviL}}$\footnote{By abuse of notation, we shall omit
  subscripts where they are thought to not be ambiguous}.  In addition
to giving each axiom, we provide a philosophical reading of what each
axiom says.  As remarked in \S\ref{validities}, \textsc{EviL} is not
\emph{normal}, that is it is not closed under variable substitution.

\begin{table}
\newcounter{rownum}
\setcounter{rownum}{0}
\newcounter{rownum2}
\setcounter{rownum2}{0}
\begin{tabularx}{\linewidth}{|cl>{\it}X|}
\hline
(\addtocounter{rownum}{1}\arabic{rownum}) & $\vdash \phi \to \psi \to \phi$ & \multirow{3}{*}{Axioms for basic propositional logic} \\
(\addtocounter{rownum}{1}\arabic{rownum}) & $\vdash (\phi \to \psi \to \chi) \to (\phi \to \psi) \to \phi \to \chi$ &  \\
(\addtocounter{rownum}{1}\arabic{rownum}) & $\vdash (\neg \phi \to \neg \psi) \to \psi \to \phi$ &  \\[6pt]
(\addtocounter{rownum}{1}\arabic{rownum}) & $\vdash \BBI_X \phi \to \phi$ & If $\phi$ holds under any further evidence $X$ considers, then $\phi$ holds simpliciter, since considering no additional evidence is trivially considering further evidence \\[6pt]
(\addtocounter{rownum}{1}\arabic{rownum}) & $\vdash \BBI_X \phi \to \BBI_X \BBI_X \phi$ & If $\phi$ holds under any further evidence $X$ considers, then $\phi$ holds whenever $X$ considers even further evidence beyond that \\[6pt]
(\addtocounter{rownum}{1}\arabic{rownum}) & $\vdash p \to \BB_X p$ & \multirow{2}{8.5cm}{Changing one's mind does not bear on matters of fact}\\
(\addtocounter{rownum}{1}\arabic{rownum}) & $\vdash p \to \BBI_X p$ & \\[6pt]
(\addtocounter{rownum}{1}\arabic{rownum}) & $\vdash \Pos_X \phi \to \BB_X \Pos_X \phi$ & The more evidence $X$ discards, the freer her imagination can run \\[6pt]
(\addtocounter{rownum}{1}\arabic{rownum}) & $\vdash \Nec_X \phi \to \Nec_X \BB_Y \phi$ &  \multirow{2}{8.5cm}{If $X$ believes a proposition, she believes it regardless of what anyone else thinks} \\
(\addtocounter{rownum}{1}\arabic{rownum}) & $\vdash \Nec_X \phi \to \Nec_X \BBI_Y \phi$ & \\[6pt]
(\addtocounter{rownum}{1}\arabic{rownum}) & $\vdash \PP_X \to \Nec_X \phi \to \phi$ & If $X$'s premises are sound, then her logical conclusion are correct \\[6pt]
(\addtocounter{rownum}{1}\arabic{rownum}) & $\vdash \PP_X \to \BB_X \PP_X $ & If $X$'s premises are sound then any subset of will be sound as well \\[6pt]
(\addtocounter{rownum}{1}\arabic{rownum}) & $\vdash \phi \to \BB_X \DDI_X \phi$ &
\multirow{2}{8.5cm}{Embracing evidence is the inverse of discarding evidence} 
\\
(\addtocounter{rownum}{1}\arabic{rownum}) & $\vdash \phi \to \BBI_X \DD_X \phi$ & \\[6pt]
(\addtocounter{rownum}{1}\arabic{rownum}) & $\vdash \Nec_X (\phi \to \psi) \to \Nec_X \phi \to \Nec_X \psi$ &
\multirow{3}{8.5cm}{Variations on axiom $K$}
\\
(\addtocounter{rownum}{1}\arabic{rownum}) & $\vdash \BB_X (\phi \to \psi) \to \BB_X \phi \to \BB_X \psi$ & \\
(\addtocounter{rownum}{1}\arabic{rownum}) & $\vdash \BBI_X (\phi \to \psi) \to \BBI_X \phi \to \BBI_X \psi$ & \\[6pt]
(\addtocounter{rownum2}{1}\Roman{rownum2}) & 
 $\AxiomC{$\vdash \phi \to \psi$}
\AxiomC{$\vdash \phi$}
\BinaryInfC{$\vdash \psi$}
\DisplayProof$ & Modus Ponens\\[10pt]
(\addtocounter{rownum2}{1}\Roman{rownum2}) & 
 $\AxiomC{$\vdash \phi$}
\UnaryInfC{$\vdash \Box_X \phi$}
\DisplayProof$ & \multirow{3}{8.5cm}{Variations on necessitation}\\
(\addtocounter{rownum2}{1}\Roman{rownum2}) & 
 $\AxiomC{$\vdash \phi$}
\UnaryInfC{$\vdash \BB_X \phi$}
\DisplayProof$ &  \\
(\addtocounter{rownum2}{1}\Roman{rownum2}) &
 $\AxiomC{$\vdash \phi$}
\UnaryInfC{$\vdash \BBI_X \phi$}
\DisplayProof$ & \\[10pt]
\hline
\end{tabularx}
\caption{A Hilbert style axiom system for \textsc{EviL}}
\label{table:axioms}
\end{table}


%%% Local Variables: 
%%% mode: latex
%%% TeX-master: "../evil_philosophy"
%%% End: 


Needless to say, these axioms shall be the focus of study for all of
our future investigations in \S\ref{army-of-darkness}.

As a final remark, we should mention that \textsc{EviL} trivially
makes true the \emph{deduction} theorem. This is because we follow
the convention set forth in \cite[Definition 4.4,
pg. 192]{blackburn_modal_2001}, and make the subsequent stipulation:
\begin{mydef}
\[ \Gamma \vdash \phi \iff  \exists \Delta \subseteq_\omega
\Gamma. \vdash \bigwedge \Delta \to \phi \]
\end{mydef}
While trivial, the deduction theorem plays a key role in the proof of
the Theorem Theorem, Theorem \ref{theorem-theorem} from
\S\ref{evil-grammar}.  Specifically, it is this definition that
justifies equation \ref{eq:deduction-theorem2}.

% We immediately have the following theorem:
% \begin{theorem}[Soundness]
% If $\vdash \phi$ then for any model $\mathfrak{M}$ and any 
% $(a,A) \in \mathfrak{M}$ we have that $\mathfrak{M},(a,A) \models \phi$ 
% \end{theorem}

% This logic makes true a variety of relationships between the various
% modalities, which are given in the following lemma:
% \begin{lemma}\label{equivs}
% We have the following provable equivalences:
% \begin{eqnarray*} \vdash \Nec_X \phi \IFF \BB_X \Nec_X \phi & \hspace{1cm} \vdash \Nec_X \phi \IFF \Nec_X \BB_Y \phi \hspace{1cm}  & \vdash \Nec_X \phi \IFF \Nec_X \BBI_Y \phi \\
%  \vdash \BB_X \phi \IFF \BB_X \BB_X \phi & \vdash \BBI_X \phi \IFF
%  \BBI_X \BBI_X \phi & \vdash \PP_X \IFF \BB_X \PP_X\end{eqnarray*}

% &\vdash \PP_X \IFF \DDI_X \PP_X \\
% &\vdash \phi \IFF \psi \textup{ implies } \vdash \chi \IFF \chi[\phi/\psi]
% \end{eqnarray*}
% \ldots where $\chi[\phi/\psi]$ is the same as $\chi$ but every occurrence of $\phi$ as a subformula is replaced by $\psi$.
% $\hfill\dashv$
% \end{lemma}


%%% Local Variables: 
%%% mode: latex
%%% TeX-master: "evil_philosophy"
%%% End: 
