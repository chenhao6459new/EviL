
In this section, we shall illustrate how we intuitively read the operators in
\tmtextsc{EviL}, and provide a number of validities.

As per the traditional doxastic reading of $\Box  \phi$, we read this as
asserting ``The \tmtextsc{EviL} agent believes $\phi$''.  Because of Theorem
\ref{theorem-theorem}, the Theorem Theorem, we shall freely conflate this with the assertion
``The \tmtextsc{EviL} agent has an argument for $\phi$,'' which we
take to be a kind of proof.

The intuition for how to read $\DM \phi$ was first mentioned in
\S\ref{Descartes} with respect to Descartes' Meditation II. It means ``If the
\tmtextsc{EviL} agent were to set aside some of her beliefs, or cast some of
her beliefs into doubt, then $\phi$ would hold.''  Dually, we can read
$\boxminus \phi$ as saying something like ``For all the ways that the
\tmtextsc{EviL} agent might use her imagination, $\phi$ holds.''  One should recognize
that these interpretations might seem inconsistent.  These are not
really an issue regard
casting beliefs into doubt and embracing one's imagination as part of the same
coin.  For, naturally, when one doubts more things, then for a
fleeting moment their dreams take flight as the inconceivable turns
around into the conceivable, if only for a little while.  To give an
example, if Marta sets aside for a moment her belief that

{\hspace*{\fill}}the law of gravity is an exceptionless regularity of the
universe, {\hspace*{\fill}}$(g)$

then it seems natural that she might imagine that

{\hspace*{\fill}}a propulsion device exploiting some exception to gravitation might be constructable.{\hspace*{\fill}}$(p)$

In the symbology of \tmtextsc{EviL} formulae, she would code this intuition as
\begin{equation} 
\boxminus (\Box  \neg g \rightarrow \diamondsuit p) . 
\end{equation}
To give another example, if Marta pretends that it is not the case that:

{\hspace*{\fill}}the canals of Amsterdam are filthy{\hspace*{\fill}}$(f)$

She might be able to imagine a scenario where

{\hspace*{\fill}}she may swim comfortably in the Amstel
river{\hspace*{\fill}}$(r)$

But not really.  Marta really cannot really swim at ease in the Amstel, not just
because it has tons of garbage, but also because

{\hspace*{\fill}}she does not own a bathing suit,{\hspace*{\fill}}$(b)$

Frankly, Marta is not so bold that she could go skinny dipping in
Amstel without that being awkward for her.  
In the language of \tmtextsc{EviL}, this thought
experiment would be expressed as follows:
\begin{equation}
\neg \boxminus (\Box  \neg f \rightarrow \diamondsuit r) \label{awkward} 
\end{equation}
This is because Marta can cast into doubt the assumption of the filthiness of the
canals of Amsterdam, while still retaining her belief that she does not have a
bathingsuit, so swimming in Amstel would still be awkward for me.  In
symbols, she would write express this other sentiment as something of
a refinement on \eqref{awkward}, which is expressed as follows:
\begin{equation} 
\DM (\diamondsuit \neg f \wedge \Box  b \wedge \neg \diamondsuit r)
\end{equation}
Further, the intuition for how to read $\DP \phi$ is ``If the
\tmtextsc{EviL} agent were to remember something, then $\phi$ would hold.'' \
For instance, imagine a scenario where Marta wakes up and searches herself
for her bike keys.  To her horror, the keys are not there -- and Marta
immediately assumes that she might have left her keys in the lock on
her bike, and figures there is a fair likelihood that

{\hspace*{\fill}}the bike has been stolen because the keys were left
in the lock.{\hspace*{\fill}}$(s)$

But once she recalls that

{\hspace*{\fill}}she lent her bike to a friend,{\hspace*{\fill}}$(l)$

her fear subsides.  Prior to remembering, while
Marta thought it might be possible that her bike was stolen due to her
own negligence,
if she remembered what she had done then she no longer would have
entertained this possibility.  This observation is expressed as:
\begin{equation} 
\diamondsuit s \wedge \boxplus (\Box  l \rightarrow \Box  \neg s)
\end{equation}
We consider $\boxminus$ and $\boxplus$ to be inverse modalities of each other,
in exactly the same way that \tmtextit{past} and \tmtextit{future} are inverse
modalities in temporal logic. This is perhaps a little unusual; it is arguably
more natural to think of \tmtextit{forgetting} as the inverse modality of
remembering, and there does not appear to be an natural inverse operation
corresponding to casting into doubt.  Following the idea of the \tmtextit{web
of belief} due to Quine, as presented in \S\ref{quine}, we would 
extend a position asserting that remembering factive data is the 
same as embracing as much of one's evidence as possible.

% In terms of the semantics outlined, $\boxminus$ corresponds to a subsetset
% relation while $\boxplus$ corresponds to a superset relation.  Because of
% this, we shall sometimes read $\boxminus \phi$ closer to the formal semantics, as
% saying something like ``for all subsets of the agent's basic beliefs
% or premises, $\phi$ holds'' and dually for $\boxplus \phi$.  
% This is admittedly even less natural than
% the reading of remembering as the opposite of casting into doubt.
% So be it; we shall have to be comfortable with \tmtextsc{EviL} 
% agents being at best twisted cartoon versions of actual people.
% Is this so unnatural?  Logic, in general, only affords at best cartoon
% versions of whatever we are trying to model with it.  Consider, for
% instance, Peano arithmetic, which is in general not powerful enough to
% prove a basic number theoretic fact like Goodstein's theorem
% \citep{kirby_accessible_1982}.  Or consider the stronger system 
% of second order arithmetic, which is in turn not strong enough to prove 
% % No logical framework for 
% % reasoning about epistemics can hope to who actually have minds and engage in remembering,
% % imagining, and other similar activities.  
% % After all, according to the
% semantics stipulated in \S\ref{evil-grammar}, \tmtextsc{EviL} agents apparently have sets
% for brains, which makes an \tmtextsc{EviL} agent a strange effigy for a person
% indeed -- with the possible exception of set theorists, whose brains are
% typically constructed entirely of sets or urelements.

% Furthermore, it is under the set theoretical reading that $\PP$ makes
% the most sense.  We should read it as asserting something like ``the basis for
% the \textsc{EviL} agent's beliefs is sound'' or ``the \textsc{EviL}
% agent's arguments only use true premises.''  It further means that the
% actual state of affairs is compatible with what the agent believes -
% reality has not been ruled out by something that the agent is taking
% as evidence.  Moreover,

Finally, reading considering the agent's web of beliefs as a network
of related sets of premises, we may draw one final philosophical
insight, which is reflective of a validity.
A sound set of premises $A$ intuitively will ensure any subset of them
is also sound.
%   Indeed, we may conclude that a set of premises is sound if and only if all of
% its subsets are also sound, which is expressed in \eqref{eq:PPiff}. 
% \begin{equation} \label{eq:PPiff}
% \PP \IFF \BM \PP
% \end{equation}

With this final intuition in mind, we turn to illustrating how
the above philosophical intuitions give rise to an assortment of
validities for \textsc{EviL}.

%%% Local Variables: 
%%% mode: latex
%%% TeX-master: "evil_philosophy"
%%% End: 
