\label{evil-intuition}
In this section, I shall illustrate how I intuitively read the operators in
\tmtextsc{EviL}, and provide a number of validities.

As per the traditional doxastic reading of $\Box  \phi$, I read this as
asserting ``The \tmtextsc{EviL} agent believes $\phi$''.  Because of Theorem
\ref{theorem-theorem}, the Theorem Theorem, I shall freely conflate this with the assertion
``The \tmtextsc{EviL} agent has an argument for $\phi$,'' which I take to be a
proof.

My intuition for how to read $\DM \phi$ was first mentioned in
\S\ref{Descartes} with respect to Descartes' Meditation II -- it means ``If the
\tmtextsc{EviL} agent were to set aside some of her beliefs, or cast some of
her beliefs into doubt, then $\phi$ would hold.''  Dually, I tend to read
$\boxminus \phi$ as saying something like ``For all the ways that the
\tmtextsc{EviL} agent might use her imagination, $\phi$ holds.''  I recognize
that these interpretations might seem inconsistent -- however, I regard
casting beliefs into doubt and embracing one's imagination as part of the same
coin.  For, naturally, when one doubts more things, then for a fleeting moment their dreams take flight as the inconceivable turns around into the conceivable, if only for a little while.  To give an example, if I set aside for a moment my belief that

{\hspace*{\fill}}the law of gravity is an exceptionless regularity of the
universe{\hspace*{\fill}}$(g)$

$\ldots$then it seems natural to imagine that

{\hspace*{\fill}}a propulsion device exploiting some exception to gravitation might be constructable.{\hspace*{\fill}}$(p)$

In the symbology of \tmtextsc{EviL} formulae, I would code this intuition as
\[ \boxminus (\Box  \neg g \rightarrow \diamondsuit p) . \]
To give another example, if I pretend that it isn't the case that:

{\hspace*{\fill}}the canals of Amsterdam are filthy{\hspace*{\fill}}$(f)$

I might be able to imagine a scenario where

{\hspace*{\fill}}I am swimming comfortably in the Amstel
river{\hspace*{\fill}}$(r)$

But not really.  I really can't really swim at ease in the Amstel, not just
because it has tons of garbage, but also because

{\hspace*{\fill}}I don't own a bathing suit,{\hspace*{\fill}}$(b)$

Frankly, I am not so bold that I could go skinny dipping in Amstel without
that being awkward.  Hence I would say in the language of \tmtextsc{EviL}
that:
\[ \neg \boxminus (\Box  \neg f \rightarrow \diamondsuit r) \]
This is because I can cast into doubt the assumption of the filthiness of the
canals of Amsterdam, while still retaining my belief that I don't have a
bathingsuit, so swimming in Amstel would still be awkward for me.  In
symbols, I would write express this sentiment as the following expression:
\[ \DM (\diamondsuit \neg f \wedge \Box  b \wedge \neg \diamondsuit r)
\]
Further, my intuition for how to read $\DP \phi$ is ``If the
\tmtextsc{EviL} agent were to remember something, then $\phi$ would hold.'' \
For instance, I can think of an instance where I woke up and searched myself
for my bike keys.  To my horror, they weren't there -- in I immediately
assumed that I might have left my keys in the lock on my bike, and figured
there was a fair likelihood that

{\hspace*{\fill}}my bike has been stolen because I left the keys in
it.{\hspace*{\fill}}$(s)$

But once I recalled that

{\hspace*{\fill}}I had lent my bike to a friend,{\hspace*{\fill}}$(l)$

$\ldots$my fear subsided.  I would have said that prior to remembering, while
I thought it might be possible that my bike was stolen due to my negligence,
if I remembered what I had done then I no longer would have entertained that
possibility.  I would express this observation as:
\[ \diamondsuit s \wedge \boxplus (\Box  l \rightarrow \Box  \neg s) \]
I consider $\boxminus$ and $\boxplus$ to be inverse modalities of each other,
in exactly the same way that \tmtextit{past} and \tmtextit{future} are inverse
modalities in temporal logic. This is perhaps a little unusual; it is arguably
more natural to think of \tmtextit{forgetting} as the inverse modality of
remembering, and there doesn't appear to be an natural inverse operation
corresponding to casting into doubt.  Following the idea of the \tmtextit{web
of belief} due to Quine, as presented in \S\ref{quine}, I would extend a position
asserting that remembering factive data is the same as embracing as much of
one's evidence as possible.

In terms of the semantics outlined, $\boxminus$ corresponds to a subsetset
relation while $\boxplus$ corresponds to a superset relation.  Because of
this, I sometimes read $\boxminus \phi$ closer to the formal semantics, as
saying something like ``for all subsets of the agent's beliefs, $\phi$ holds''
and dually for $\boxplus \phi$.  This is admittedly even less natural than
the reading of remembering as the opposite of casting into doubt.  So be it;
I am comfortable with \tmtextsc{EviL} agents being at best twisted cartoon
versions of actual people, who actually have minds and engage in remembering,
imagining, and other similar activities.  After all, according to the
semantics stipulated in \S\ref{evil-grammar}, \tmtextsc{EviL} agents apparently have sets
for brains, which makes an \tmtextsc{EviL} agent a strange effigy for a person
indeed -- with the possible exception of set theorists, whose brains are
typically constructed entirely of sets or urelements.

Furthermore, it is under the set theoretical reading that $\PP$ makes
the most sense.  I read it as asserting something like ``the basis for
the \textsc{EviL} agent's beliefs is sound'' or ``the \textsc{EviL}
agent's arguments only use true premises.''  It further means that the
actual state of affairs is compatible with what the agent believes -
reality has not been ruled out by something that the agent is taking
as evidence.  Moreover, sound premises intuitively exhibit the
following property - any subset of them is also sound, since soundness
isn't a phenomenon that is subject to synchronicity or other failures
of compositionality.  A set of premises is sound if and only if all of
its subsets are also sound.

%%% Local Variables: 
%%% mode: latex
%%% TeX-master: "evil_philosophy"
%%% End: 
