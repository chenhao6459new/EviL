In this section, we shall present an alternative work to the framework
proposed in \S\ref{explicit}. 
These semantics are inspired by game semantics for modal logic, 
such as those in \citet[chapter 2]{van_benthem_modal_2010}.

First, recall the basic modal grammar $\mathcal{L}_K(\Phi)$:
\[ \phi ::= p \in \Phi \ |\ \phi \to \psi \ | \ \bot \ |\ \Box \phi \]

Next, consider structures of the form $\langle W, V, \beta, \iota \rangle$
consisting of:
\begin{itemizedot}
  \item A set of worlds $W$
  \item A propositional valuation function $V : \Phi \rightarrow \powerset
  W$
  \item An belief function $\beta : W \rightarrow \powerset \mathcal{L}_K
  (\Phi)$
   \item An imagination function $\iota : W \rightarrow \powerset W$
\end{itemizedot}
We shall call these \tmtextit{belief-imagination models}.   One can think of a
model $\Omega$ sort of like a of tuples like in \S\ref{evil-semantics}; however
in this case evidently it would have to be $\Omega \subseteq \powerset
\Phi \times \powerset \mathcal{L}_K (\Phi) \times \powerset \Omega$, so
apparently it would have to be a non-wellfounded set.   This is somewhat natural,
given a modal logic setting - see for instance \citep{barwise_vicious_1996} for an
elaboration on these connections.

\begin{definition}
  Define by recursion the following two truth relations:
  
  First relation:
  \begin{empt}
    \item $\Omega, w \Vdash p \Longleftrightarrow p \in V (w)$
    \item $\Omega, w \Vdash \phi \wedge \psi \Longleftrightarrow$ both $\Omega,
    w \Vdash \phi$ and $\Omega, w \Vdash \psi$
    \item $\Omega, w \Vdash \phi \vee \psi \Longleftrightarrow$ either $\Omega,
    w \Vdash \phi$ or $\Omega, w \Vdash \psi$
    \item $\Omega, w \Vdash \neg \phi \Longleftrightarrow \Omega, w \Vvdash
    \phi$
    \item $\Omega, w \Vdash \Box \phi \Longleftrightarrow \beta (w)
    \vdash^{\ast} \phi$
    
    Where $\vdash^{\ast}$ is a sequent that is closed under reflection
    and resolution:
    \begin{eqnarray*}
      \frac{\phi \in \Gamma}{\Gamma \vdash^{\ast} \phi} &  & \frac{\Gamma
      \vdash^{\ast} \neg \phi \vee \psi \ \ \  \Delta \vdash^{\ast}
      \phi}{\Gamma \cup \Delta \vdash^{\ast} \psi}
    \end{eqnarray*}
  \end{empt}
  Second relation:
  \begin{empt}
    \item $\Omega, w \Vvdash p \Longleftrightarrow p \nin V (w)$
    \item $\Omega, w \Vvdash \phi \wedge \psi \Longleftrightarrow$ either $\Omega, w \Vvdash \phi$ or $\Omega, w \Vvdash \psi$
    \item $\Omega, w \Vvdash \phi \vee \psi \Longleftrightarrow$ both $\Omega,
    w \Vvdash \phi$ and $\Omega, w \Vvdash \psi$
    \item $\Omega, w \Vvdash \neg \phi \Longleftrightarrow \Omega, w \Vdash
    \phi$
    \item $\Omega, w \Vvdash \Box \phi \Longleftrightarrow$ there is some $v
    \in \iota (w)$ such that $\Omega, v \Vvdash \phi$
  \end{empt}
\end{definition}

It is necessary to motivate the intuition behind these semantics.  
Informally, we think of these two truth relations correspond to two
players, whom we shall call the
{\tmem{logician}} and the {\tmem{philosopher}}.   The logician wields a set
beliefs given by $\beta$ and tries to compose compelling arguments, and the
philosopher employs a corpus of thought experiments given by $\iota$ to thwart
the logician's arguments.   Of course, the logician and the philosopher are
really just two aspects of a single epistemic agent we are trying to model; we
shall imagine epistemic agents modeled by this system to be embroiled in internal
conflict.  This sort of dissension between reason and imagination rages
on within us all -- it is fundamental to human nature.

These semantics are not naturally bivalent; that is it does not hold that
either $\Omega, w \Vdash \phi$ or $\Omega, w \Vvdash \phi$, exclusively. To
see this consider a model where $\beta (w) = \iota (w) = \varnothing$; then
evidently $\Omega, w {\nVdash} \Box p$ and $\Omega, w {\nVvdash} \Box p$.

However, bivalence has a convenient semantic characterization:

\begin{proposition}
  \label{biv1}Let $\mathbbm{M}^{\Omega} = \langle W^{\Omega}, V^{\Omega},
  R^{\Omega} \rangle$ be a model for basic modal logic model based on a
  belief/imagination model $\Omega$, where $w R^{\Omega} v \assign v \in \iota
  (w)$, and let $\Vdash_{\Box}$ be the modal truth predicate.   We have that
  $\Vdash$ and $\Vvdash$ are bivalent if and only if $\Omega, w \Vdash \phi
  \Longleftrightarrow \mathbbm{M}^{\Omega}, w \Vdash_{\Box} \phi$.
\end{proposition}

\begin{proof}
  $(\Longrightarrow)$  Assume that $\Vdash$ and $\Vvdash$ are bivalent and
  consider any $\phi \in \mathcal{L}_K (\Phi)$.   The proof that $\Omega, w
  \Vdash \phi$ is equivalent to $\mathbbm{M}^{\Omega}, w \Vdash_{\Box} \phi$
  proceeds by induction.   The case for proposition letters, conjunction and
  disjunction are straightforward, so we shall only consider negation and
  modality.
  
  Negation:  We have the following chain of equivalences:
  \begin{eqnarray*}
    \mathbbm{M}^{\Omega}, w \Vdash_{\Box} \neg \phi  
    \Longleftrightarrow & \mathbbm{M}^{\Omega}, w {\nVdash}_{\Box} \phi &
    \\
    \Longleftrightarrow & \Omega, w {\nVdash} \phi & (\tmop{inductive}
    \tmop{step})\\
    \Longleftrightarrow & \Omega, w \Vvdash \phi & (\tmop{bivalence})\\
    \Longleftrightarrow & \Omega, w \Vdash \neg \phi & 
  \end{eqnarray*}
  Modality: \ We have another chain of equivalences:
  \begin{eqnarray*}
    \Omega, w \Vdash \Box \phi \  \Longleftrightarrow & \Omega, w
    {\nVvdash} \Box \phi & (\tmop{bivalence})\\
    \Longleftrightarrow & \forall v \in \iota (w) .  \Omega, w {\nVvdash}
    \phi & (\tmop{definition})\\
    \Longleftrightarrow & \forall v \in \iota (w) .  \Omega, w \Vdash \phi &
    (\tmop{bivalence})\\
    \Longleftrightarrow & \forall v \in \iota (w) .\mathbbm{M}^{\Omega}, w
    \Vdash_{\Box} \phi & (\tmop{inductive} \tmop{step})\\
    \Longleftrightarrow & \forall v.w R^{\Omega} v \Longrightarrow
    \mathbbm{M}^{\Omega}, w \Vdash_{\Box} \phi & (\tmop{definition})\\
    \Longleftrightarrow & \mathbbm{M}^{\Omega}, w \Vdash_{\Box} \Box \phi & 
  \end{eqnarray*}
  This completes the induction.
  
  $(\Longleftarrow)$ \ Assume that $\Omega, w \Vdash \phi$ and
  $\mathbbm{M}^{\Omega}, w \Vdash_{\Box} \phi$ are always equivalent.   We
  have:
  \begin{eqnarray*}
    \Omega, w \Vdash \phi \  \Longleftrightarrow &
    \mathbbm{M}^{\Omega}, w \Vdash_{\Box} \phi & \\
    \Longleftrightarrow & \Omega, w {\nVdash}_{\Box} \neg \phi & \\
    \Longleftrightarrow & \Omega, w {\nVdash} \neg \phi &
    (\tmop{hypothesis})\\
    \Longleftrightarrow & \Omega, w {\nVvdash} \phi & 
  \end{eqnarray*}
\end{proof}

\begin{corollary}
  If $\Vdash$ and $\Vvdash$ are bivalent, then $\beta (w) \vdash^{\ast} \phi$
  for all $\phi \in \tmop{Th}_{\Vdash} (\Omega)$ for all $w \in W^{\Omega}$,
  where $\tmop{Th}_{\Vdash} (\Omega) =\{\phi \in \mathcal{L}_K (\Phi)
  \  | \  \Omega, w \Vdash \phi$ for all $w \in W^{\Omega}
  \}$.
\end{corollary}

Evidently bivalence of $\Vdash$ and $\Vvdash$ gives rise to semantics where
the agent has a proof for every proposition they believe.   Furthermore, we
can take any modal logic model $\mathbbm{M} \assign \langle W^{\mathbbm{M}},
V^{\mathbbm{M}}, R^{\mathbbm{M}} \rangle$ and define an equivalent
belief/imagination model $\Omega^{\mathbbm{M}} \assign \langle
W^{\mathbbm{M}}, V^{\mathbbm{M}}, \beta^{\mathbbm{M}}, \iota^{\mathbbm{M}}
\rangle$ where:
\begin{eqnarray}
  \beta^{\mathbbm{M}} (w) \assign & \{\phi \in \mathcal{L}_K (\Phi) \ 
  | \  \mathbbm{M}, w \Vdash_{\Box} \Box \phi\} &  \nonumber\\
  \iota^{\mathbbm{M}} (w) \assign & \{v \in W^{\mathbbm{M}} \  |
  \  w R^{\mathbbm{M}} v\} &  \nonumber
\end{eqnarray}
We can immediately leverage this to give the a characterization of these
semantics:

\begin{proposition}
  The basic modal logic $K$ is sound and strongly complete for bivalent
  belief/imagination models.
\end{proposition}

\begin{proof}
  Soundness is trivial given the previous lemma, strong completeness follows
  by considering the canonical model $\mathbbm{K}$ and looking at
  $\Omega^{\mathbbm{K}}$. 
\end{proof}

However, recalling on the remarks presented in \S\ref{thermometers},
it is wrong for agents to be able to have everything they believe in 
their minds; this is about as bad as the thermometer theory of knowledge.   
However, this is evidently not entirely necessary. 
Call a belief/imagination model
\tmtextit{reasonable} if the following two constraints are satisfied:
\begin{itemizedot}
  \item $\beta (w) \vdash^{\ast} \phi$ for all $\phi \in \tmop{Th}_{\Vdash}
  (\Omega)$ for all $w \in W^{\Omega}$, where $\tmop{Th}_{\Vdash} (\Omega)
  =\{\phi \in \mathcal{L}_K (\Phi) \  | \  \Omega, w \Vdash
  \phi$ for all $w \in W^{\Omega} \}$
  
  \item $\tmop{Mod}^{\Omega}_{{\nVvdash}} (\beta (w)) \subseteq \iota (w)$,
  where $\tmop{Mod}^{\Omega}_{{\nVvdash}} (\beta (w)) =\{v \in W^{\Omega}
  \  | \  \Omega, v {\nVvdash} \phi$ for all $\phi \in
  \beta (w)$\}
  
  \item $\beta (w) \backslash \tmop{Th}_{\Vdash} (\Omega)$ is finite
\end{itemizedot}
Evidently, forcing these requirements suffices to force bivalence:

\begin{proposition}
  Let $\mathbbm{M}^{\Omega}$ be defined as in Prop.  \ref{biv1}.
  For any reasonable model $\Omega$ and any $w \in W^{\Omega}$, we have:
  \begin{enumerateroman}
    \item If $\mathbbm{M}^{\Omega}, w \Vdash_{\Box} \phi$ then $\Omega, w
    \Vdash \phi$
    
    \item If $\mathbbm{M}^{\Omega}, w {\nVdash}_{\Box} \phi$ then $\Omega,
    w \Vvdash \phi$
  \end{enumerateroman}  
  Hence we have $\Vdash$ and $\Vvdash$ are bivalent.
\end{proposition}

\begin{proof}
  The propositional, disjunctive and conjunctive cases are all
  straightforward; we shall focus on negation and modality.
  
  Negation: In the case of (\tmtextit{i}), we know that \
  \begin{eqnarray}
    \mathbbm{M}^{\Omega}, w \Vdash_{\Box} \neg \phi \Longleftrightarrow &
    \mathbbm{M}^{\Omega}, w {\nVdash}_{\Box} \phi &  \nonumber\\
    \Longrightarrow & \Omega, w \Vvdash \phi & \text{(by the inductive step)}
    \nonumber\\
    \Longleftrightarrow & \Omega, w \Vdash \neg \phi &  \nonumber
  \end{eqnarray}
  The proof for (\tmtextit{ii}) is similar.
  
  Modality: \ In the case of (\tmtextit{i}), assume that
  $\mathbbm{M}^{\Omega}, w \Vdash_{\Box} \Box \phi$. Using the definition of
  reasonableness and the inductive step we know for all $v \in W^{\Omega}$
  that if $\Omega, v {\nVvdash} \psi$ for all $\psi \in \beta (w)
  \backslash \tmop{Th} (\Omega)$ then $\Omega, v \Vdash \phi$.
  
  From this and the fact that $\Omega$ is reasonable we can infer that
  $\bigvee_{\psi \in \beta (w) \backslash \tmop{Th} (\Omega)} \neg \psi \vee
  \phi \in \tmop{Th}_{\Vdash} (\Omega)$.   We know further from reasonableness
  that we have $\tmop{Th}_{\Vdash} (\Omega) \subseteq \beta (w)$. So we can
  prove by induction that repeatedly applying resolution gets $\beta (w)
  \vdash^{\ast} \phi$, which just means that $\Omega, w \Vdash \Box \phi$, as
  desired.
  
  The case of (\tmtextit{ii}) follows trivially by induction.
\end{proof}

We may continue to obtain weak completeness for these semantics:

\begin{proposition}
  $\vdash_K \phi$ if and only if $\Omega, w \Vdash \phi$ for all reasonable
  models $\Omega$ and $w \in W^{\Omega}$
\end{proposition}

\begin{proof}
  Left to right follows straightforwardly, so we just need to prove right to
  left.
  
  Assume $\nvdash_K \phi$. As before, let $\mathbbm{M} = \langle
  W^{\mathbbm{M}}, V^{\mathbbm{M}}, R^{\mathbbm{M}} \rangle$ be a finite model
  and with a world $w \in W^{\mathbbm{M}}$ such that $\mathbbm{M}, w
  {\nVdash}_{\Box} \phi$.   Now consider a slightly modified model
  $\mathbbm{M}' \assign \langle W^{\mathbbm{M}}, V', R^{\mathbbm{M}} \rangle$
  where
  \[ V' (p) \assign \left\{ \begin{array}{ll}
       \{v\} & p = \rho (v)\\
       V (p) & o / w
     \end{array} \right.  \]
  A proof by induction on subformulae $\psi$ of $\phi$ verifies that
  $\mathbbm{M}, w \Vdash_{\Box} \psi$ if and only if $\mathbbm{M}', w
  \Vdash_{\Box} \psi$.
  
  So now consider $\Omega \assign \langle W^{\mathbbm{M}'}, V^{\mathbbm{M}'},
  \tau, \lambda x.R^{\mathbbm{M}'} [x] \rangle$ such that
  \[ \tau (w) \assign \tmop{Th} (\mathbbm{M}') \cup \left\{ \bigvee_{v \in
     R^{\mathbbm{M}'} [w]} \rho (v) \right\}, \]
  where $\tmop{Th} (\mathbbm{M}') \assign \{\psi \in \mathcal{L}_K
  (\Phi) \  | \  \mathbbm{M}', v \Vdash \psi$ for all $v
  \in W^{\mathbbm{M}'} \}$.   A proof by induction on $\psi$ shows that
  $\mathbbm{M}', w \Vdash_{\Box} \psi$, $\Omega, w \Vdash \psi$ and $\Omega, v
  {\nVvdash} \psi$ are equivalent for all $\psi \in \mathcal{L}_K (\Phi)$.  
  Thus we have that for all $v \in W^{\Omega}$ that $\Omega, v {\nVvdash}
  \psi$ for all $\psi \in \tmop{Th} (\mathbbm{M}')$.   Moreover, evidently $w
  R^{\mathbbm{M}'} v$ if and only if $\mathbbm{M}', v \Vdash_{\Box} \bigvee_{u
  \in R^{\mathbbm{M}'} [w]} \rho (u)$, whence we have that $w R^{\mathbbm{M}'}
  v$ if and only if $\Omega, v {\nVvdash} \chi$ for all $\chi \in \tau
  (w)$. With this we can employ induction and establish that $\mathbbm{M}', w
  \Vdash_{\Box} \psi$ if and only if $\Omega, w \models \psi$ for all $\psi
  \in \mathcal{L}_K (\Phi)$.   Since $\mathbbm{M}', w {\nVdash}_{\Box} \phi$,
  we have that $\Omega, w {\nmodels} \phi$.   Finally, note that in this
  model we have that $\tmop{Mod}^{\Omega}_{{\nVvdash}} (\beta (w)) =
  R^{\mathbbm{M}'} [w]$. With this and the definition of $\Omega$, we can see
  that $\Omega$ is evidently reasonable, and thus we may complete the proof.

\end{proof}

Now, while reasonable models attain the goal of modeling agents that have
proofs for the things they believe, they should not be considered adequate. These models are only reasonable in the
sense that they indeed model agents providing nontrivial proofs for their beliefs.
However, they are not reasonable in the sense that they are simple
to reckon with.   So while the semantics provided in
\S\ref{evil-semantics} requires a grammar restriction, it should be
preferred over the formulation given above,
precisely because it is more manageable.