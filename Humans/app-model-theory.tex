Recall that \ref{pV}, presented in Prop. \ref{evil_models} in
\S\ref{kripke} states:
\begin{proposition}
  For any \textsc{EviL} model $\Omega$,  $\mho^{\Omega}$ has the
  following property:

 \hfil $(R^{\Omega}_X \circ \sqsubseteq^{\Omega}_X) \subseteq
    R^{\Omega}_X \subseteq (R^{\Omega}_X \circ \sqsupseteq^{\Omega}_X)$
    \hfil \ref{pV}
\end{proposition}

In other words, if $(a,A) R^{\Omega}_X (c,C)$ and $(a,A)
\sqsupseteq^{\Omega}_X (b,B)$, then $(b,B) \sqsupseteq^{\Omega}_X (c,C)$

Recall that along with this principle, the following philosophical
reading was offered:

\begin{center}
``If the agent assumes fewer things, more things are imaginable,
since it's easier for a world to be incompatible with an agent's evidence.''
\end{center}

In fact, in light of Theorem \ref{theorem-theorem}, the Theorem
Theorem, \ref{pV} follows from a general model theoretic relationship.
For a given Kripke structure $\mathbbm{M}$, define two operators $Mod^{\mathbbm{M}}
: \powerset  \mathcal{L} (\Phi, \mathcal{A}) \rightarrow \powerset
(W^{\mathbbm{M}})$ and $Th^{\mathbbm{M}} : \powerset
(W^{\mathbbm{M}}) \rightarrow \powerset \mathcal{L} (\Phi, \mathcal{A})$
\begin{align*}
  Mod^{{\mathbbm{M}}}({\Delta}) &
  =\{x{\in}W{\ }|{\ }{\forall}{\psi}{\in}{\Delta}.
  {\mathbbm{M}},x{\Vdash}{\psi}\}\\
  Th^{{\mathbbm{M}}}({\nabla}) & =\{{\psi}{\in}\mathcal{L}({\Phi},
  \mathcal{A}){\ }|{\ }{\forall}x{\in}{\nabla}.
  {\mathbbm{M}},x{\Vdash}{\psi}\}
\end{align*}
We then have, for any $\Delta \in \powerset  \mathcal{L}
(\Phi, \mathcal{A})$ and $\nabla \in \powerset (W^{\mathbbm{M}})$:
\[ \nabla \subseteq Mod^{\mathbbm{M}} (\Delta) \text{ if and only if }
   \Delta \subseteq Th^{\mathbbm{M}} (\nabla) \]
$\ldots$hence we these two operations form what is refered an
\emph{antitone Galois connection}, between the lattice $\powerset
(W^{\mathbbm{M}})$ and the lattice $\powerset \mathcal{L} (\Phi,
\mathcal{A})$. It follows from the theory of Galois connections
\citep[][chapter 3]{roman_lattices_2008} that
the following two properties:
\begin{align}
  \text{If $\nabla \supseteq \nabla'$ then $Th^{\mathbbm{M}} (\nabla)
  \subseteq Th^{\mathbbm{M}} (\nabla')$} \\
  \text{If $\Delta \supseteq \Delta'$ then $Mod^{\mathbbm{M}} (\Delta)
  \subseteq Mod^{\mathbbm{M}} (\Delta')$} \label{mod-antitone}
\end{align}

We can see that \ref{pV} follows from \eqref{mod-antitone}. 
To see this, assume that $(a, A) \sqsupseteq_X^{\Omega} (b, B)$.  Then observe:
\begin{align*}
(a, A) \sqsupseteq_X^{\Omega} (b, B) & \ \implies a = b \textup{ and } A_X
\supseteq B_X & \textup{by the definition of $\sqsupseteq_X^{\Omega}$} \\
& \ \implies A_X \supseteq B_X & \textup{weakening} \\
& \ \implies Mod^{\Omega}(A_X) \subseteq Mod^{\Omega}(B_X) &
\textup{from \eqref{mod-antitone}} \\
& \ \implies \textup{if } \Omega,(c,C)\models A_X\textup{ then }
\Omega,(c,C)\models B_X &
\textup{by the definition of $Mod^{\Omega}$} \\
& \ \implies \textup{if } (a, A) R^{\Omega}_X (c,C)\textup{ then }
(b, B) R^{\Omega}_X (c,C)  &
\textup{by the definition of $R_X^\Omega$} \\
\end{align*}

The above line of reasoning illustrates that structural 
features of \textsc{EviL} models are consequences of
the decision to set $\Omega,(a,A) \VDash \Box \phi \iff Th(\Omega)
\cup A \vdash \phi$.

%  hence $A_X
% \supseteq B_X$ and thus $Mod^{\Omega} (A_X) \subseteq
% Mod^{\Omega} (B_X)$.  But it follows from semantics of \tmtextsc{EviL}
% we have that $(c, C) \in Mod^{\Omega} (A_X)$ if and only if $(a, A)
% R^{\Omega}_X (c, C)$, and likewise for $B_X$.  Hence if $(a, A) R^{\Omega}_X
% (c, C)$ then $(b, B) R^{\Omega}_X (c, C)$.

%%% Local Variables: 
%%% mode: latex
%%% TeX-master: "evil_philosophy"
%%% End: 
