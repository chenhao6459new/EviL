% \documentclass{letter}
% \usepackage{geometry,amsmath,amssymb}
% \geometry{letterpaper}
% 
% %%%%%%%%%% Start TeXmacs macros
% \newcommand{\assign}{:=}
% \newcommand{\colons}{\,:\,}
% \newcommand{\tmop}[1]{\ensuremath{\operatorname{#1}}}
% \newcommand{\tmtextsc}[1]{{\scshape{#1}}}
% \newenvironment{itemizedot}{\begin{itemize} \renewcommand{\labelitemi}{$\bullet$}\renewcommand{\labelitemii}{$\bullet$}\renewcommand{\labelitemiii}{$\bullet$}\renewcommand{\labelitemiv}{$\bullet$}}{\end{itemize}}
% \newenvironment{proof}{\noindent\textbf{Proof\ }}{\hspace*{\fill}$\Box$\medskip}
% \newtheorem{definition}{Definition}
% \newtheorem{lemma}{Lemma}
% \newtheorem{theorem}{Theorem}
% %%%%%%%%%% End TeXmacs macros

%\begin{document}

In section \S\ref{validities}, I presented the structural validities of
\text{EviL} from a philosophical perspective.  That being the case, my
manner of presentation followed my intuition, which I
admit is altogether unorganized.  In this section, I shall give the
validities of \textsc{EviL} a more systematic presentation.  In doing
so, I shall showcase an elimination theorem, that I feel sits at the
heart of \textsc{EviL}.

%In this section, I illustrate that the structural validities of
%\textsc{EviL} can be used to reason that $\BM, \BP, \DM,$ and $\DP$
%exhibit certain behavior over two dual fragments of the language
%$\mathcal{L}(\Phi)$.

To start, the following lemma summerizes the structural validities
that I will be studying in the subsequent discussion:

\begin{lemma}
  \label{validities}The following validities hold for all \tmtextsc{EviL}
  models:
  \[ \begin{array}{lll}
       \VDash \boxminus p \leftrightarrow p & \hspace{5em} & \VDash \boxplus p
       \leftrightarrow p\\
       \VDash \boxminus \neg p \leftrightarrow \neg p &  & \VDash \boxplus
       \neg p \leftrightarrow \neg p\\
       \VDash \boxminus \diamondsuit \phi \leftrightarrow \diamondsuit \phi & 
       & \VDash \boxplus \Box \phi \leftrightarrow \Box \phi\\
       \VDash \DM \Box \phi \leftrightarrow \Box \phi &  &
       \VDash \DP \diamondsuit \phi \leftrightarrow \diamondsuit
       \phi\\
       \VDash \boxminus \boxminus \phi \leftrightarrow \boxminus \phi &  &
       \VDash \boxplus \boxplus \phi \leftrightarrow \boxplus \phi\\
       \VDash \boxminus \DP \phi \leftrightarrow \DP \phi &  & \VDash \boxplus \DM \phi
       \leftrightarrow \DM \phi\\
       \VDash \boxminus \circlearrowleft \leftrightarrow \circlearrowleft &  &
       \VDash \boxplus \neg \circlearrowleft \leftrightarrow \neg
       \circlearrowleft
     \end{array} \]
\end{lemma}

These validities suggest a definite interplay between the modalities of
\tmtextsc{EviL}; they are highly suggestive of a general elimination
theorem.  To see what arises from Lemma \ref{validities}, first
observe that \textsc{EviL} makes true the usual substitution rule:

\begin{lemma}
  If $\VDash \phi \leftrightarrow \psi$ is a validity, then $\VDash \chi
  \leftrightarrow \chi [\phi / \psi]$ is a validity for any $\chi \in
  \mathcal{L} (\Phi)$.
\end{lemma}

Next, I offer two sublanguages of the main language of
\tmtextsc{EviL}:

\begin{definition}
  Define the following fragments:\footnote{The fragment $\mathcal{L}_A(\Phi)$ was inspired from the continuous fragment of $\mu \tmop{PML}$ (FIXME).}
  
  $\mathcal{L}_A (\Phi)$:
  \[ \phi {::=} \  p \  | \  \neg p
     \  | \  \top \  | \  \bot
     \  | \  \circlearrowleft \  | \ 
     \phi \wedge \psi \  | \  \phi \vee \psi \ 
     | \  \diamondsuit \phi \  | \  \boxminus
     \phi \  | \  \DP \phi \]
  $\mathcal{L}_B (\Phi)$:
  \[ \phi {::=} \  \neg p \  | \  p
     \  | \  \bot \  | \  \top
     \  | \  \neg \circlearrowleft \  |
     \  \phi \vee \psi \  | \  \phi \wedge \psi
     \  | \  \Box \phi \  | \  \DM \phi \  | \  \boxplus \phi \  \]
\end{definition}

\begin{definition}
  Define two dualizing operations $(\cdot)^A : \mathcal{L}_B (\Phi)
  \rightarrow \mathcal{L}_A (\Phi)$ and $(\cdot)^B : \mathcal{L}_A (\Phi)
  \rightarrow \mathcal{L}_B (\Phi)$, using recursion, such that:
  \[ \begin{array}{ll}
       \neg p^A \assign p & p^B \assign \neg p\\
       p^A \assign \neg p & \neg p^B \assign p\\
       \bot^A \assign \top & \top^B \assign \bot\\
       \top^A \assign \bot & \bot^B \assign \top\\
       \neg \circlearrowleft^A \assign \circlearrowleft & \circlearrowleft^B
       \assign \neg \circlearrowleft\\
       (\phi \vee \psi)^A \assign \phi^A \wedge \psi^A & (\phi \wedge \psi)^B
       \assign \phi^B \vee \psi^B\\
       (\phi \wedge \psi)^A \assign \phi^A \vee \psi^A & (\phi \vee \psi)^B
       \assign \phi^B \wedge \psi^B\\
       (\Box \psi)^A \assign \diamondsuit (\psi^A) & (\diamondsuit \psi)^B
       \assign \Box(\psi^B)\\
       (\DM \psi)^A \assign \boxminus (\psi^A) & (\boxminus
       \psi)^B \assign \DM (\psi^B)\\
       (\boxplus \psi)^A \assign \DP (\psi^A) & (\DP \psi)^B \assign \boxplus (\psi^B)
     \end{array} \]
\end{definition}

With the above definition in hand, it is straightforward to see the following duality
theorem:

\begin{theorem}[Duality]
  Observe that for all $\phi \in \mathcal{L}_A (\Phi)$ and $\psi \in
  \mathcal{L}_B (\Phi)$, $(\phi^B)^A = \phi$ and $(\psi^A)^B = \psi$. \
  Moreover, we have the following validities: $\VDash \neg (\phi^B)
  \leftrightarrow \phi$ and $\VDash \neg (\psi^A) \leftrightarrow \psi$.
\end{theorem}

The above duality is convenient, since it can be leveraged to transfer
results proven for one the fragment $\mathcal{L}_A (\Phi)$ to $\mathcal{L}_B (\Phi)$.

With the above machinery in place, I present what I feel is the
natural consequence of the logical equivalence given in
Lemma \ref{validities}:

\begin{definition}
  If $\phi \in \mathcal{L}_X (\Phi)$ then let $\phi^{\ast}$ be the same
  formula, with all instances of $\boxplus$, $\boxminus$, $\DM$
  and $\DP$ eliminated.
\end{definition}

\begin{theorem}[\textsc{EviL} Elimination]\label{vanishing}
  For all $\phi \in \mathcal{L}_A (\Phi)$ or $\phi \in \mathcal{L}_B (\Phi)$,
  we have the following validity:
  \[ \VDash \phi \leftrightarrow \phi^{\ast} \]
\end{theorem}

\begin{proof}
  The prove proceeds in three steps.
  
  Step 1: First, use induction on $\phi \in \mathcal{L}_A (\Phi)$, and show
  the following two facts simultaneously:
  \begin{align*}
    {\VDash}{\boxminus}{\phi}{\leftrightarrow}{\phi} &
    {\hspace{3em}}{\VDash}\DP{\phi}{\leftrightarrow}{\phi}
  \end{align*}
  \begin{itemizedot}
    \item Cases $p$, $\neg p$, $\bot$, $\top$, $\circlearrowleft$: \ In all of
    these situations, the result follows directly from the validities illustrated in
    Lemma \ref{validities}.
    
    \item Cases $\wedge, \vee$: \ For $\boxminus$ the connective $\wedge$ is
    simple, and dually for $\DP$ for the connective $\vee$. \
    This is because in each case one may simply use distribution, such as can
    be done here:
    \begin{align*}
      {\VDash}{\boxminus}({\phi}{\wedge}{\psi}) &
      {\leftrightarrow}{\boxminus}{\phi}{\wedge}{\boxminus}{\psi}\\
      & {\leftrightarrow}{\phi}{\wedge}{\psi}
    \end{align*}
    
    On the other hand, $\vee$ is more interesting for $\boxminus$, and dually
    $\wedge$ for $\DP$.  Using induction, Lemma
    \ref{validities}, and subsitution, and distribution, we have the line of
    reasoning:
    \begin{align*}
      {\VDash}{\boxminus}({\phi}{\vee}{\psi}) &
      {\leftrightarrow}{\boxminus}(\DP{\phi}{\vee}\DP{\psi})\\
      & {\leftrightarrow}{\boxminus}\DP({\phi}{\vee}{\psi})\\
      & {\leftrightarrow}\DP({\phi}{\vee}{\psi})\\
      &
      {\leftrightarrow}\DP{\phi}{\vee}\DP{\psi}\\
      & {\leftrightarrow}{\phi}{\vee}{\psi}
    \end{align*}
    
    \item Case $\diamondsuit$: Once again, this follows immediately from the
    validities of Lemma \ref{validities}, namely $\VDash \boxminus
    \diamondsuit \phi \leftrightarrow \diamondsuit \phi$ and $\VDash \DP \diamondsuit \phi \leftrightarrow \diamondsuit \phi$
    
    \item Cases $\boxminus, \DP$: The final step follows from
    one more application of Lemma \ref{validities}, namely by employing the
    following four validities
    \[ \begin{array}{ll}
         \VDash \boxminus \DP \phi \leftrightarrow \DP \phi & \VDash \DP \DP \phi
         \leftrightarrow \DP \phi\\
         \VDash \boxminus \boxminus \phi \leftrightarrow \boxminus \phi &
         \VDash \DP \boxminus \phi \leftrightarrow \boxminus
         \phi
       \end{array} \]
  \end{itemizedot}
  Step 2: \ With the above, we can prove for any $\phi \in \mathcal{L}_A
  (\Phi)$ that $\VDash \phi \leftrightarrow \phi^{\ast}$.  Once again, the
  proof proceeds by induction, the only steps worth noting involve $\boxminus$
  and $\DP$.  In either case, employing the step can be
  completed using Step 1. For instance, we know that $\VDash \boxminus \phi
  \leftrightarrow \phi$, hence $\VDash \boxminus \phi \leftrightarrow
  \phi^{\ast}$ by induction.
  
  Step 3: \ With the result for $\mathcal{L}_A (\Phi)$ in hand, just observe
  that for $\psi \in \mathcal{L}_B (\Phi)$ we have that $(\psi^A)^{\ast} =
  (\psi^{\ast})^A$. With this, substitution, and duality, we have the
  following chain of reasoning:
  \begin{align*}
    {\VDash}{\psi} & {\leftrightarrow}{\neg}({\psi}^A)\\
    & {\leftrightarrow}{\neg}(({\psi}^A)^{{\ast}})\\
    & {\leftrightarrow}{\neg}(({\psi}^{{\ast}})^A)\\
    & {\leftrightarrow}{\neg}({\neg}((({\psi}^{{\ast}})^A)^B))\\
    & {\leftrightarrow}{\neg}{\neg}{\psi}^{{\ast}}\\
    & {\leftrightarrow}{\psi}^{{\ast}}
  \end{align*}
\end{proof}

\begin{example}
 The following validities of \textsc{EviL} are consequences of Theorem \ref{vanishing}:
\[\VDash \BM\DP\Pos\DP\BM\Pos\Pos\DP\BM\DP\Pos\BM\Pos\BM\DP\BM\Pos\DP\neg p  \IFF \Pos\Pos\Pos\Pos\Pos\Pos\neg p \]
\[ \VDash ((\DM\BP\Nec q\wedge\BP\DM\Nec q)\vee\Nec\BP\DM q)\wedge((\BP\Nec\DM q\vee\Nec\DM\BP q)\wedge\DM\Nec\BP q)  \IFF \Nec q \]
\[ \VDash \BM\DP\Pos t \vee \DM\BP\Box \neg t \]
\end{example}

The way I read Theorem \ref{vanishing} is that $\boxminus$
and $\DP$ are empty modalities on $\mathcal{L}_A (\Phi)$, and dually for $\mathcal{L}_B
(\Phi)$ with $\DM$ and $\boxplus$.  Further, note that
$\mathcal{L}_0 (\Phi) \subseteq \mathcal{L}_A (\Phi) \cap \mathcal{L}_B
(\Phi)$ (up to translation), which means that all four of $\boxplus$,
$\boxminus$ along with their duals $\DP$ and $\DM$
vanish on the propositional language.  Inspecting the semantics, this is to
be expected, since neither $\boxplus$ nor $\boxminus$ interact with
propositional truth values.

Finally, I should remark that Theorem \ref{vanishing} reflects one of
the basic themes of \textsc{EviL} - the interplay between belief,
reflected by $\Box$, and
imagination, reflected by $\Pos$.  I feel that these two phenomena are just two sides of
the same coin - furthermore, one couldn't have more natural opposites.
Belief and imagination exemplify what I naturally feel are two warring
forces dwelling within any \textsc{EviL} agent's heart.  Evidently
soundness $\PP$ is aligned with imagination and
unsoundness $\neg\PP$ is aligned with belief.  However, I admit that I
do not understand what philosophical light is shed by Theorem
\ref{vanishing}, if there is indeed any at all.

% \end{document}

%%% Local Variables: 
%%% mode: latex
%%% TeX-master: "evil_philosophy"
%%% End: 
