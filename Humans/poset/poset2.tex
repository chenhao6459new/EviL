% Rule based diagram
% Author: Remus Mihail Prunescu
\documentclass[]{article}
%\usepackage{pxfonts}            % Or palatino or mathpazo
%\usepackage{eulervm}            %

\usepackage{tikz}
\usetikzlibrary{matrix,positioning,shadows,arrows}
\usepackage[graphics,tightpage,active]{preview}
\PreviewEnvironment{tikzpicture}
\newlength{\imagewidth}
\newlength{\imagescale}

\newcommand{\seq}{\Rightarrow} 

\begin{document}
\begin{center}
\begin{tikzpicture}
  \matrix (m) [matrix of math nodes, row sep=3em,
  column sep=3em, text height=1.5ex, text depth=0.25ex]
  {
  |(B)| &             & |(13)| \circ &             & |(12)| \circ &              & |(15)|  & \\
  &             & |(10)| \circ & |(9)| \circ &              &  |(11)| \circ  \\
  & |(8)| \circ & |(6)| \circ &              &  |(7)| \circ &             \\
  & |(3)| \circ & |(2)| \circ &              & |(4)| \circ  & |(5)| \circ \\
  |(A)| & |(0)| \circ &             & |(1)| \circ  &              &             &  |(14)|  & \\ };
  \path[densely dotted]
  (1) edge [bend left=30] node[below,sloped] {} (3)
  (1) edge [bend right=30] node[below,sloped] {} (5)
  (5) edge [bend right=30] node[above,sloped] {} (9)
  (0) edge [bend right=30] node[below,sloped] {} (9)
  (4) edge [bend right=30] node[below,sloped] {} (10)
  (0) edge node[above,sloped] {} (3)
  (6) edge node[above,sloped] {} (10)
  (10) edge node[above,sloped] {} (13)
  (9) edge node[above,sloped] {} (12)
  (11) edge node[above,sloped] {} (12)
  (6) edge [bend left=30] node[above,sloped] {} (12)
  (7) edge node[above,sloped] {} (9)
  (7) edge node[above,sloped] {} (11)
  (1) edge node[above,sloped] {} (2)
  (1) edge node[above,sloped] {} (4)
  (3) edge node[above,sloped] {} (6)
  (3) edge node[above,sloped] {} (8)
  (2) edge node[above,sloped] {} (6)
  (4) edge node[above,sloped] {} (7)
  (14) edge[solid, ->, >=latex, line width=8]node[below,sloped] {Premises} (15)
  (A) edge[solid, ->, >=latex, line width=8]node[above,sloped] {Experiential Data} (B);
\end{tikzpicture}
\end{center}
\end{document}