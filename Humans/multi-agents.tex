In this section we extend the semantics for
\tmtextsc{EviL} from a single agent, as presented in \S\ref{evil-grammar}, to
accommodate multiple agents.  This is primarily of interest since further
results in \textsc{EviL}, namely completeness, can naturally be
abstracted beyond the single agent case.  
% But I will freely admit that
% my \textsc{EviL} intuitions are principally grounded in the single
% agent case -- I recommend thinking about the multi-agent case as just
% a generalization of the single agent case.

The following provides the definition of the language of multi-agent \textsc{EviL}:
\begin{definition} Define $\mathcal{L} (\Phi, \mathcal{A})$ by the following Backus-Naur grammar:
\[ \phi \ {::=} \  p \in \Phi \  | \  \phi
   \rightarrow \psi \  | \  \bot \  |
   \  \Box_X \phi \  | \  \boxminus_X \phi
   \  | \  \boxplus_X \phi \  | \ 
   \circlearrowleft_X \]
Here $X \in \mathcal{A}$, and $\mathcal{A}$ is non-empty.
\end{definition}

As in the single agent case, multi-agent \tmtextsc{EviL} models are 
sets $\Omega
\subseteq \powerset \Phi \times   (\powerset
\mathcal{L}_0 (\Phi))^\mathcal{A}$ -- that is, $\Omega$ is a set of
pairs of sets of proposition letters, and indexed sets of
propositional formulae.
 
The semantic entailment relation for
multi-agent \textsc{EviL} is 
\[ (\VDash) \colons \powerset ( \powerset \Phi \times  
   (\powerset \mathcal{L}_0 (\Phi))^\mathcal{A}) \times \powerset \Phi
   \times ( \powerset \mathcal{L}_0 (\Phi))^\mathcal{A}
   \times \mathcal{L} (\Phi, \mathcal{A}) \rightarrow
   \textup{\textsf{bool}}. \]
The input/output behavior of $(\VDash)$ is just as it was defined before in
\S\ref{evil-grammar}, the only difference in this setting is that instead of
taking a pair as an input, where the second element is a
set, it takes an indexed set.

We now provide a formal definition of the semantics for the multi-agent $(\VDash)$:
{\footnote{Where $X \in \mathcal{A}$, 
we shall use $A_X$ to denote $A (X)$ provided that $A \colons \mathcal{A} \rightarrow
\powerset \mathcal{L}_0 (\Phi)$}}
\begin{definition}
\begin{align*}
  {\Omega},(a,A){\VDash} p & {\iff}p{\in}a\\
  {\Omega},(a,A){\VDash} {\phi}{\rightarrow}{\psi} &
  {\iff}{\Omega},(a,A){\VDash}{\phi}\text{ implies
  }{\Omega},(a,A){\VDash}{\psi}\\
  {\Omega},(a,A){\VDash}{\bot} & {\iff} False\\
  {\Omega},(a,A){\VDash}\Box_X {\phi} & {\iff}{\forall}(b,B){\in}{\Omega}.
  ({\forall}{\psi}{\in}A_X. b{\models}{\psi})\text{ implies
  }{\Omega},(b,B){\VDash}{\phi}\\
  {\Omega},(a,A){\VDash}{\boxminus}_X{\phi} &
  {\iff}{\forall}(b,B){\in}{\Omega}. a=b\text{ and }B_X{\subseteq}A_X\text{
  implies }{\Omega},(b,B){\VDash}{\phi}\\
  {\Omega},(a,A){\VDash}{\boxplus}_X{\phi} &
  {\iff}{\forall}(b,B){\in}{\Omega}. a=b\text{ and }B_X{\supseteq}A_X\text{
  implies }{\Omega},(b,B){\VDash}{\phi}\\
  {\Omega},(a,A){\VDash}{\circlearrowleft}_X & {\iff}
  {\forall}{\psi}{\in}A_X.a{\models}{\psi}
\end{align*}
\end{definition}

Just as in \S\ref{evil-grammar}, Lemma \ref{truthiness} and Theorem
\ref{theorem-theorem} can be seen to obtain for the new generalized
semantics.  Furthermore, all of the validities mentioned in \S\ref{validities}
and \S\ref{elimination} hold, along with Theorem \ref{vanishing}, 
where $\Box$, $\Pos$, $\BM$,$\BP$, $\DM$, $\DP$ and $\PP$ 
are all replaced with $\Box_X$, $\Pos_X$, $\BM_X$, $\BP_X$,
$\DM_X$, $\DP_X$ and $\PP_X$ respectively, for any fixed 
$X \in \mathcal{A}$.
% Furthermore, compactness still fails, as shall be  presented in 
% \S\ref{non-compactness}.

Finally, there are two novel validities that arise in these semantics:
\begin{eqnarray*} 
& \VDash \Box_X \phi \to \Box_X \BM_Y \phi \\
& \VDash \Box_X \phi \to \Box_X \BP_Y \phi 
 \end{eqnarray*}

This is just to say, that the \textsc{EviL} agent's deliberative
process is opaque to other's beliefs, just as in the single agent
case. This was expressed by
\eqref{dontcare1} and \eqref{dontcare2} in \S\ref{validities}. 
The agent cannot read anyone else's mind, nor anyone
else hers.

% I will admit that little use shall be made of \textsc{EviL} generalized
% this way, since the single agent case is far more natural to think
% about for me.  

Using the multi-agent semantics we have developed here, 
the proof theory for \textsc{EviL} that shall be presented in 
\S\ref{army-of-darkness} can now be given in higher generality.

%%% Local Variables: 
%%% mode: latex
%%% TeX-master: "evil_philosophy"
%%% End: 
