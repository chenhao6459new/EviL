In this section we provide definitions and lemmas related to the
subformula construction $\Cross^\phi$.  We follow
\cite[chapter 5, pgs. 78--84]{boolos_logic_1995} in our approach, 
as well as the
``Fischer-Ladner Construction'' used in the completeness theorem of PDL
\cite[chapter 4, pgs. 241--248]{blackburn_modal_2001}.

We first recall the definition of \emph{pseudo-negation} from the
Fischer-Ladner construction for the completeness of PDL \cite[chapter
4, pgs. 243]{blackburn_modal_2001}.  We shall also introduce
\emph{pseudo-boxes}, which are defined as follows:
\begin{mydef}
\begin{eqnarray*}
\sim \phi := \begin{cases} \psi &\textup{ if $\phi = \neg\psi$} \\ \neg \phi & \textup{ o/w} \end{cases} &
\hspace{1cm} \pBB_X \phi := \begin{cases} \phi &\textup{ if $\phi = \BB_X \psi$} \\ \BB_X \phi & \textup{ o/w} \end{cases} \hspace{1cm} &
\pBBI_X \phi := \begin{cases} \phi &\textup{ if $\phi = \BBI_X \psi$} \\ \BBI_X \phi & \textup{ o/w} \end{cases}
\end{eqnarray*}
\end{mydef}
Like pseudo-negation, the idea of pseudo-boxes
is that they raise the semantic behavior of operators to the syntactic
level.  This is summarized in the following lemma:
\begin{lemma}\label{equivs2} 
\begin{eqnarray*} 
\vdash \sim \phi \IFF \neg \phi &
\hspace{1cm} \vdash \pBB_X \phi \IFF \BB_X \phi \hspace{1cm} &
\vdash \pBBI_X \phi \IFF \BBI_X \phi
\end{eqnarray*}
\begin{align*}
\pBB_X \phi = \pBB_X\pBB_X \phi & & \pBBI_X\phi = \pBBI_X \pBBI_X \phi \hfill
\end{align*}
\end{lemma}
\begin{proof}
We remind the reader that $\vdash$ here abbreviates $\vdash_\textsc{EviL}$.
\begin{description}
  \item[$\vdash \sim \phi \IFF \neg \phi$ --] Assume that $\phi$ is
    unnegated, then $\sim \phi = \neg \phi$ and hence we know that
    $\vdash \neg \phi \IFF \neg \phi$, which suffices.  If $\phi =
    \neg \psi$, then we know that $\sim \phi = \psi$, and since in
    classical logic we have that $\vdash \psi \IFF \neg \neg \psi$ we
    have the result.
  \item[$\vdash \pBB_X \phi \IFF \BB_X \phi$ --] If $\phi$ is not
    boxed with $\BB_x$, the result is trivial.  So assume that $\phi =
    \BB_X \psi$, then $\pBB_X \phi = \BB_X \psi$. 
   
   Note that for 
    \textsc{EviL} Kripke structures, for which $\BB_X$ corresponds to
    $\sqsupseteq_X$, then from \textsc{EviL}ness we know that
    $\sqsupseteq_X$ is transitive and reflexive, hence
    $\Vdash_{\textsc{EviL}} \BB_X \psi \IFF \BB_X \BB_X \psi$.  By
    completeness, we know that $\vdash \BB_X \psi \IFF \BB_X \BB_X
    \psi$.  But this suffices exactly what we wanted to prove.

  \item[$\vdash \pBP_X \phi \IFF \BP_X \phi$ --] This result follows
    using exactly the same reasoning as above, only it uses the fact
    that the \emph{dual} of $\sqsupseteq_X$, which is $\sqsubseteq_X$, is
    reflexive and transitive too.

  \item[$\pBM_X \phi = \pBM_X \pBM_X \phi$ --] First assume that $\phi$
     is a $\BM_X$ boxed formula.  Then 
   $\pBM_X \phi = \pBM_X \pBM_X \phi = \phi$.  Next assume that $\phi$
   is not a $\BM_X$ boxed formula, then $\pBM_X \phi = \BM_X \phi$,
   and hence 
\begin{align*}
\pBM_X \pBM_X \phi & = \pBM_X \BM_X \phi \\
& = \BM_X \phi \\
& = \pBM_X \phi
\end{align*}
  \item[$\pBP_X \phi = \pBP_X \pBP_X \phi$ --] The proof of this
    assertion is exactly the same as the proof for $\pBM_X$.
\end{description}
\end{proof}

We shall use these operations above in the subformula construction we
will carry out.  Next, we introduce an operation which will 
allow us to restrict our subformulae to
precisely the finitary number of agents that shall be relevant.

\begin{mydef} Let $\delta(\phi) \subseteq \mathcal{A}$
% , ``the \textbf{dudes} mentioned by $\phi$'', 
be the set of agents that  occur in $\phi$\end{mydef}

We now employ primitive recursion to define the finite set of formulae
that we shall use in our construction, which we have labeled
$\Sigma$.  This operation behaves as follows:
\begin{bul}
\item $\Sigma$ takes as input:
\begin{bul}
  \item A set of agents $\Delta$
   \item A formula $\phi$ where $\phi \in \mathcal{L}(\mathcal{A},\Phi)$ 
\end{bul}
\item $Sigma$ outputs a set $S$ of $\mathcal{L}(\mathcal{A},\Phi)$
  formulae (that is, $S \subseteq \mathcal{L}(\mathcal{A},\Phi)$)
\end{bul}
We may summarize this concisely as the following type signature:
\[ \Sigma : (\powerset \mathcal{A}) \times
\mathcal{L}(\mathcal{A},\Phi) \to \powerset
\mathcal{L}(\mathcal{A},\Phi) \]

\begin{mydef}\label{EviLSigma}Define $\Sigma( \Delta,\phi)$ using primitive 
recursion as follows:
\begin{align*}
\Sigma(\Delta,p) := &   \{ p, \neg p, \bot, \neg\bot \} \cup \\
& \{ \BB_X p, \neg\BB_X p, \BBI_X p, \neg\BBI_X p  \ |\ X \in \Delta
\} \\
\Sigma(\Delta,\bot) :=  & \{ \bot, \neg\bot \} \\
% \Sigma(\Delta,\PP_X) :=  & \{ \PP_X, \neg\PP_X, \BB_X \PP_X, \neg
% \BB_X \PP_X, \bot, \neg\bot \} \\
\Sigma(\Delta,\PP_X) :=  & \{ \PP_X, \neg\PP_X, \BB_X \PP_X, \neg
\BB_X \PP_X, \bot, \neg\bot \} \\
\Sigma(\Delta,\phi\to\psi) := & \{ \phi\to\psi,\neg(\phi\to\psi) \}
\cup   \Sigma(\Delta,\phi) \cup \Sigma(\Delta,\psi)\\
\Sigma(\Delta,\Nec_X \phi) := & \{ \Nec_X \phi, \neg \Nec_X \phi,
\BBI_X \Nec_X \phi, \neg\BBI_X\Nec_X \phi \} \cup \\
& \{\Nec_X \pBB_Y \phi, \neg\Nec_X \pBB_Y \phi, \\
&\ \; \Nec_X \pBBI_Y \phi, \neg\Nec_X \pBBI_Y \phi, \\
&\ \; \pBB_Y \phi, \neg \pBB_Y \phi, \\
&\ \; \pBBI_Y \phi, \neg\pBBI_Y \phi\ |\ Y \in \Delta \} \cup \\
& \Sigma(\Delta,\phi) \\
\Sigma(\Delta,\BB_X \phi) := & \{ \BB_X \phi, \neg\BB_X \phi\} \cup \Sigma(\Delta,\phi) \\
\Sigma(\Delta,\BBI_X \phi) := & \{ \BBI_X \phi, \neg\BBI_X \phi\} \cup \Sigma(\Delta,\phi)
\end{align*}
\end{mydef}

To understand how the above operates, we assume that the reader has
some background in recursive programming.  Recall how ``subformulae''
are defined for the Fischer-Ladner construction for the completeness
proof of PDL. We can see that authors describe the set of subformulae
$\neg FL(\Sigma)$ as follows:
\begin{quote}  We defined $\neg FL(\Sigma)$, the \emph{closure of
    $\Sigma$}, as the smallest set containing which is Fischer-Ladner
  closed and closed under single negations \cite[pg. 243]{blackburn_modal_2001}.
\end{quote}
Here Fischer-Ladner closed means the construction satisfies certain
subformula properties, such as ``if $\langle \pi_1 ; \pi_2 \rangle
\phi \in \neg FL(\Sigma)$ then $\langle \pi_1 \rangle \langle \pi_2
\rangle \phi \in \neg FL(\Sigma)$.'' We ask the reader who knows a
little about computers, how would one go about programming 
the Fischer-Ladner closure?  
The easiest way to program the Fischer-Ladner closure, in
languages like \textsf{Haskell} or \textsf{OCaml}, would
be to use pattern recognition and (primitive) recursion.  
This is just as we have done, informally, for the \textsc{EviL} subformulae construction.

We argue that writing a concise, programmatic recursive
characterization as we have done for $\Sigma$ is the easiest way to
express the set with the features we desire. 
For one thing, the closure properties we shall
want depend at the top-level on a constant set of agents,
which are calculated at the beginning of the construction.  
Moreover, as we shall see, we shall need some formulae boxed
in certain ways to ensure certain partly \textsc{EviL} properties and
certain formulae boxed in other ways to ensure other partly \textsc{EviL}
properties.  Worse yet -- since we have multiple kinds of pseudo
operators, we cannot enforce closure for all of them.  Managing
the priorities of when a formula should be closed for which operations
roughly amounts to giving the algorithmic characterization we have
carried out above.

In our subsequent proofs, we shall capitalize on combinatoric
properties that our subformulae operation obeys.  Some of these
features are summarized in the following proposition.  
% The proof of these
% assertions is offered as an exercise to our readers. 
%, who we hope are
%rather fond of inductive proofs and sudoku-like activities:

\begin{proposition}\label{inclusions}
$\Sigma(\delta(\phi),\phi)$ is finite.  Moreover, we have the following:
\begin{bul}
        \item $\phi \in \Sigma(\delta(\phi),\phi)$
	\item If $\psi \in \Sigma(\delta(\phi),\phi)$ and $\chi$ is a subformula of $\psi$, then $\chi \in \Sigma(\delta(\phi),\phi)$
	\item If $\psi \in \Sigma(\delta(\phi),\phi)$ then $\sim \psi \in \Sigma(\delta(\phi),\phi)$
	\item If $\BB_X \phi \in \Sigma(\delta(\phi),\phi)$ then $\pBB_X \phi \in \Sigma(\delta(\phi),\phi)$
	\item If $\BBI_X \phi \in \Sigma(\delta(\phi),\phi)$ then $\pBBI_X \phi \in \Sigma(\delta(\phi),\phi)$
\end{bul}
\end{proposition}

We follow \cite[pg. 243]{blackburn_modal_2001} in our definition of
the set of (relativized) maximally consistent sets:

\begin{mydef}[Atoms] Let $At(\Psi)$ denote the maximally consistent subsets of $\Psi$
\end{mydef}
We next have the Lindenbaum Lemma:
\begin{lemma}[Lindenbaum Lemma]If $\Gamma \nvdash \phi$ and $\Gamma\subseteq \Sigma(\delta(\phi),\phi)$, then there is a $\gamma \in At(\Sigma(\delta(\phi),\phi))$ such that $\Gamma \subseteq \gamma$ and $\gamma \nvdash \phi$
\end{lemma}
\begin{proof}
The proof of this assertion follows the same proof of the 
finitary Lindenbaum Lemma offered in \cite[Lemma 4.83, pg. 244]{blackburn_modal_2001} and \cite[pgs. 79]{boolos_logic_1995}.
\end{proof}

We now turn to defining the \textsc{EviL} subformula model we shall
use in the subsequent finitary completeness theorem:

\begin{mydef}
Define 
$$\Cross^\phi := \langle W, R, \sqsubseteq, \sqsupseteq, V, P \rangle$$
Where:
\begin{align*}
W := & At(\Sigma(\delta(\phi),\phi)) \\
V(p) := & \{ w \in W \ |\ p \in w \} \\
P_X := & \{ w \in W \ |\ \PP_X \in w \} \cup \{ w\in W \ | \ X \nin \delta(A)\}\\
R_X := & \{(w,v) \in W \times W \ |\ \{ \psi\ |\ \Box_X \psi \in w\}
\subseteq v \} \\
\sqsupseteq_X  := & \begin{cases} \{(w,v) \in W \times W \ |\
  \{\psi,\pBB_X \psi \ |\ \pBB_X \psi \in w\} \subseteq v\ \& &
  \multirow{2}{*}{\textup{when $X \in \delta(\phi)$}} \\
 \hspace{3cm}\ \;\;  \{\psi,\pBBI_X \psi \ |\ \pBBI_X \psi \in v\}
 \subseteq w \} \} \\
\{ (w,w) \ |\ w \in W\} & \textup{o/w}
\end{cases} 
\\
\sqsubseteq_X  := & \begin{cases} \{(v,w) \in W \times W \ |\
  \{\psi,\pBB_X \psi \ |\ \pBB_X \psi \in w\} \subseteq v\ \& &
  \multirow{2}{*}{\textup{when $X \in \delta(\phi)$}} \\
 \hspace{3cm}\ \;\;  \{\psi,\pBBI_X \psi \ |\ \pBBI_X \psi \in v\}
 \subseteq w \} \} \\
\{ (w,w) \ |\ w \in W\} & \textup{o/w}
\end{cases} 
\end{align*}
\end{mydef}

In the above construction, we note that the definition of $V(p)$,
$P_X$ and $R_X$ are defined as usual.  Only $\sqsupseteq_X$ and
$\sqsubseteq_X$ have are unusual; we are consciously imitating the
completeness techniques given in \cite[chapter 5,
pgs. 78--84]{boolos_logic_1995} for \textsc{EviL}.

We shall now show that $\Cross^\phi$ satisfies the \emph{Truth lemma}.
Once again, the method of the proof of the following theorem is adapted
from \cite[chapter 5, pgs. 78--84]{boolos_logic_1995}.

\begin{lemma}[Truth Lemma]\label{truth}
For any subformula $\psi \in \Sigma(\delta(\phi),\phi)$ and any $w \in
W^{\Cross}$, we have that 
\[\Cross^\phi, w \Vdash \psi \iff \psi \in w\]
\end{lemma}
\begin{proof} The proof proceeds by induction on $\psi$.
\begin{description}
\item[$p \in \Phi$, $\PP_X$, $\bot$ --] These steps are elementary.
\item[$\phi \to \psi$ --] Since we know that $\Sigma(\delta(\phi),\phi)$
   is closed under subformulae, from the inductive hypothesis we have
   that $\Cross^\phi, w \Vdash \phi \iff \phi \in w$ and $\Cross^\phi,
   w \Vdash \phi \iff \phi \in w$.  The rest of the step involves
   reasoning by cases, using the fact that $\Sigma(\delta(\phi),\phi)$ is
   closed under pseudo-negation, $w$ is maximal and pseudo-negation 
   logically equivalent to negation.
\item[$\Box_X\phi$ --]  
  The right to left direction follows by the fact
  that $\Sigma(\delta(\phi),\phi)$ is closed under subformulae, and
  the inductive hypothesis.  Hence we shall concern ourselves with the
  left to right direction.

So assume that $\Box_X \psi \nin w$ we shall show that there is a $v$
such that $w R_x v$ and $\Cross^\phi, v \nVdash \psi$.  Consider the
set 
\[ \Xi := \{ \sim \psi \} \cup \{\chi \ |\ \Box_X \chi \in w\}\]
Note that $\Xi \subseteq \Sigma(\delta(\phi),\phi)$.  If $\Xi$ is consistent, then $\Xi \nvdash \phi$ and we know from the
Lindenbaum Lemma that $\Xi$ may be extended to the $v$ we desire.  

So suppose towards a contradiction that $\Xi$ is not consistent.
Then $\vdash \neg \bigwedge \Xi$, which means by classical logic that: 
$$\vdash \left(\bigwedge_{\Box_X \chi \in w} \chi\right) \to \psi$$
But then we know by modal logic that:
$$\vdash \left(\bigwedge_{\Box_X \chi \in w} \Box_X \chi\right) \to \Box_X \psi$$
This means that since $w \vdash \bigwedge_{\Box_X \chi \in w} \Box_X
\chi$, then we have that $\Box_X \psi \in w$ by maximality after all. This is
ridiculous! $\lightning$

\item[$\BM_X\phi$ --]  This case is similar to the case for $\Box_X \phi$,
  but harder to understand.  

We shall demonstrate the left to right direction, since right to left
is elementary.  
Assume that $\BB_X \psi \nin w$, then we shall find a $v$ such that $w
\sqsupseteq_x v$ and $\Cross^\phi, v \nVdash \phi$.  Since $w$ is
maximal and $\vdash \pBM_X\psi \IFF \BM_X \psi$, we have that $w
\nvdash \pBB_X \psi$.  

Now abbreviate:
\begin{align*}
A := &  \{\chi,\pBB_X \chi \ |\ \pBB_X \chi \in w \}\\
B := & \{ \sim \pBBI_X \chi \ | \ \pBBI_X \chi \in \Sigma(\delta(\phi),\phi)\ \&\ \sim\chi \in w \}
\end{align*}
As before, if $\{\sim \psi\} \cup A \cup B$ consistent then it extends
to the desired $v$.

So suppose towards a contradiction that $\{\sim \psi\} \cup A \cup B
\vdash \bot$.  Then $A \cup B \vdash \psi$, and furthermore by the
equivalences in Lemma \ref{equivs2} and rule \eqref{BMnec} and the
axioms we have that\footnote{Here $\pBB_X S$ is shorthand for $\{
  \pBB_X \chi \ |\ \chi \in S\}$.} $$\pBB_X A \cup \pBB_X B \vdash
\pBB_X \psi.$$ So let 
\begin{align*}
A' := & \{ \pBB_X \chi\ |\ \pBB_X \chi \in w \}\\
B' := & \{ \sim \chi \ | \  \sim\chi \in w \}
\end{align*}
Since $\pBB_X \pBB_X \chi = \pBB_X \chi$ by Lemma \ref{equivs2}, we have $A' = \pBB_X A$.  Moreover, by Lemma \ref{equivs2}, axiom \eqref{reverseAx1}, and classical logic we can see that
\[ \vdash \sim \chi \to \pBB_X \sim \pBBI_X \chi \]
Thus for every $\beta \in \pBB_X B$ we have that $B' \vdash \beta$.  Hence by $n$ applications of the Cut rule we can arrive at 
\[ A' \cup B' \vdash \pBB_X \chi \]
However, evidently $A' \cup B' \subseteq w$, hence $w \vdash \pBB_X
\psi$, which is a contradiction! $\lightning$

To complete the argument, we must show that $w \sqsupseteq_X
v$.  Since $A \subseteq v$ we just need to check that $\{\psi,\pBBI_X \psi \ |\ \pBBI_X \psi \in v\} \subseteq w$.  Suppose
that $\pBBI_X \psi \in v$ but $\psi \nin w$.  Since $w$ is maximally
consistent we have then that $\sim \psi \in w$, hence $ \sim \pBP_X
\in B$ by definition and thus $\sim \pBBI_X
\psi \in v$, since $B \subseteq v$.  This contradicts that $v$ is consistent. $\lightning$

Now suppose that $\pBBI_X \psi \in v$ but $\pBBI_X \psi \nin w$, hence
$\sim \pBBI_X \psi \in w$ and thus $\sim \pBBI_X \pBBI_X \psi \in v$.
However we know from Lemma \ref{equivs2} that $\pBBI_X \pBBI_X \psi =
\pBBI_X \psi$, which once again implies that $v$ is
inconsistent. $\lightning$
\item[$\BP_X\phi$ --]  This is exactly as in the case of $\BM_X\phi$,
  only the appropriate dual assertions of all of the statements used 
are employed.
\end{description}
\end{proof}

We shall now turn to establishing that our finite model is indeed a
partly \textsc{EviL} Kripke structure, following the manner that we
used to show the same result for the
canonical \textsc{EviL} model $\mathscr{E}$.

\begin{lemma}[$\Cross^\phi$ is Partly \textsc{EviL}]\label{cross-is-partly}
$\Cross^\phi$ is a finite partly \textsc{EviL} Kripke structure.
\end{lemma}
\begin{proof}
The fact that $\Cross^\phi$ is finite follows from the fact that
$W^\Cross \subseteq \Sigma(\delta(\phi),\phi)$ and
$\Sigma(\delta(\phi),\phi)$ is itself finite, as we established in Lemma \ref{equivs2}.

The remainder of the proof is devoted to establishing the partly
\textsc{EviL} properties for $\Cross^\phi$.
  \begin{description}
    \item[\ref{ppI}]We must show ``$\sqsubseteq_X$ is reflexive.''
We first note that if $X \nin \delta(\phi)$, then this is true immediately by the
construction of $\Cross^\phi$. 

 So assume that $X \in \delta(\phi)$. We need to show two facts:
\begin{eqnarray*} 
&   \{\psi,\pBB_X \psi \ |\ \pBB_X \psi \in w\} \subseteq w \\
 &  \& \\
&\{\psi,\pBBI_X \psi \ |\ \pBBI_X \psi \in w\}
\subseteq w
\end{eqnarray*}

However, these facts follow from Lemma \ref{equivs2}, the maximality
of $w$ and the fact that we know \textsc{EviL} logic proves:
\begin{eqnarray*}
& \vdash \BB_X \psi \to \psi\\
&\textup{ and } \\
& \vdash \BBI_X \psi \to \psi
\end{eqnarray*}

We know that \textsc{EviL} proves these facts from our
previous completeness theorem, Theorem \ref{evil-completeness}, and 
the fact that $\sqsubseteq_X$ is reflexive for \textsc{EviL} models.

\item[\ref{pptrans}] A quick glance at the definition of
      $\Cross^\phi$ reveals that ``$\sqsubseteq_X$ is transitive''
is immediate by construction.

    \item[\ref{ppreverse}]
Once again, the assertion
``$\sqsubseteq_X$ is the reverse $\sqsupseteq_X$'' 
follows immediately by construction.

    \item[\ref{islandiff}] Assume $w \sqsubseteq_X v$, we shall show
      that 
$$w \in V (p)\textup{ if and only if }v \in V (p)$$

If $X \nin \delta(\phi)$ then we know that $w = v$, so the above is true.

     So assume that $X \in \delta(\phi)$.  
    Now assume that $w \in V(p)$, then $\mathscr{E}, w \Vdash p$.
     This means that $p \in w$, whence $p \in
     \Sigma(\delta(\phi),\phi)$.  
    By Definition \ref{EviLSigma} we have that 
    $\BP_X p = \pBP_X p \in \Sigma(\delta(\phi),\phi)$. 
     By axiom \eqref{letterAx2} and 
     the Truth Lemma (Lemma \ref{truth}),  we have that $\pBP_X
      p \in w$, whence $p \in v$ by construction of $\Cross^\phi$.  The other direction
      is similar, however it uses axiom \eqref{letterAx1} instead.

    \item[\ref{ppV}] 
To prove the assertion
$$ (R_X \circ \sqsubseteq_X) \subseteq
    R_X \subseteq (R_X \circ \sqsupseteq_X)$$
We first note that if $X\nin\delta(\phi)$, then $R_X \circ
\sqsubseteq_X = R_X$ by construction.

So assume $X \in \delta(\phi)$.  
We shall show only show that 
$$ (R_X \circ \sqsubseteq_X) \subseteq R_X $$
Since the other inclusion follows from \ref{ppreverse}, which we have
previously established.

So assume that $w \sqsubseteq_X u$ and $u R_X v$, we need to show that
$w R_X v$.  In order to do this, by construction we need to show that
if $\Box_X \phi \in w$ then $\phi \in v$. However, we know that
$\vdash \Box_X \phi \to \pBP_X \Box_X \phi$ by the logic of
\textsc{EviL}, and by the definition of \ref{EviLSigma} we know
that if $\Box_X \phi \in \Sigma(\delta(\phi),\phi)$ then $\pBP_X\Box_X
\phi \in \Sigma(\delta(\phi),\phi)$ too, so $\pBP_X \Box_X \phi \in w$
by maximality.  However, we then have that $\Box_X \phi \in u$ by
construction, and thus $\phi \in v$ as desired. 

    \item[\ref{ppIX}] We must show ``If $w \in P_X$ and $w \sqsupseteq_X
      v$ then $v \in P_X$.''  If $w \in P_x$ then by construction we
      know that $\PP_X \in w$.  Note that this can only happen if $X
      \in \delta(\phi)$.  As in the case of \ref{islandiff} we
      know by the definition of $\Sigma$, \textsc{EviL} logic 
      and maximality we have that $ \BM_X \PP_X \in w$.  
      Whence we have $\PP_X \in v$ as desired. 

     \item[\ref{ppantisym}] Just as in the proof of
       \ref{partly-evil-completeness} from
       \S\ref{partly-evil-strong-soundness-and-completeness}, 
      we have intentionally deferred the proof that
      $\sqsubseteq$ is a partial order, since it depends on the above
      results.  As before, that it is reflexive and transitive by
      \ref{ppI} and \ref{pptrans}.  
       Moreover, the exact same inductive proof, along with the 
      Truth Lemma, may be used to establish anti-symmetry.

    \item[\ref{ppVI}] We must show:
    $$(\sqsubseteq_Y \circ R_X) = R_X = (\sqsupseteq_Y \circ R_X)$$
Given \ref{ppI}, we know that 
$$ (\sqsubseteq_Y \circ R_X) \supseteq R_X \subseteq (\sqsupseteq_Y
\circ R_X) $$
So we shall show that $\sqsubseteq_Y \circ R_X \subseteq R_X$, since
$\sqsupseteq_Y$ is analogous.

First, observe that we may assume that $Y \in \delta(\phi)$, 
since if not then $\sqsubseteq_Y \circ R_X = R_X$ as we want.  
Now assume that $\Nec_X \phi \in w$, $w R_X v$ and $v \sqsubseteq_Y
u$; we must show that $\phi \in u$.  
Since $Y \in \delta(\phi)$ then by the construction of $\Sigma$, 
along with the \textsc{EviL} fact that 
$\vdash \Nec_X \phi \to \Nec_X \pBP_X\phi $, we know that 
$Nec_X \pBP_Y \phi \in w$, whence $pBP_Y \phi \in v$.  Since $v
\sqsubseteq_Y u$, we have that $\phi \in u$ as desired.
%corresponds to axioms \eqref{islandDown} and \eqref{islandUp}.

    \item[\ref{ppVII}] To show ``If $w \in P (X)$ then $w R_X w$,''
    assume $\PP_X \in w$ and $\Nec_X \phi \in w$.  Then we know by
    \textsc{EviL} and maximality that $\phi \in w$, which suffices to
    show that $w R_X w$. 
\end{description}
\end{proof}

We may combine the results above with what we have shown previously in
\S\ref{completely-evil} to give the following series of results:

\begin{theorem}[Abstract Finite \textsc{EviL} Soundness and Weak
  Completeness]\label{abst-finite-completeness}\ 
\begin{center}
\textsc{EviL} is weakly sound and complete for finite \textsc{EviL}
Kripke structures.
\end{center}
\end{theorem}
\begin{proof}
Since soundness is straightforward, we only prove completeness.

Assume that $\nvdash \phi$, in other words we have $\varnothing \vdash
\phi$.  This means by the Lindenbaum Lemma that $\varnothing$ may be
extended to some $w \in At(\delta(\phi),\phi)$ such that 
$\Cross^\phi,w\nVdash \phi$.  We know that $\Cross^\phi$ is partly
\textsc{EviL} from Theorem \ref{cross-is-partly}, so from Theorem
\ref{EviL-Completion} we have that $\invis^\Cross$ is an \textsc{EviL}
model, and since it is bisimilar to $\Cross$ we know that
$\invis^\Cross, w_l \nVdash \phi$.
Now note that for $\invis$ obeys the following rule:  If $|W|$ is
finite then $|W^\invis| = 2 \times |W|$ (since all that $\invis$ is
doing is making duplicates of all of the worlds).  Hence we know that
$\invis^\Cross$ is indeed finite, which means it is a suitable witness.
\end{proof}

\begin{theorem}[\textsc{EviL} Small Model Property]\label{small-model-property}\ 
If $\phi$ is satisfiable by some \textsc{EviL} Kripke structure, then $\phi$ is
satisfied by some finite \textsc{EviL} Kripke structure $\mathbb{M}$ where
$|\mathbb{M}| \in O(EXP2(|\phi|))$
\end{theorem}
\begin{proof}
Assume that $\phi$ is satisfiable in some \textsc{EviL} Kripke
structure, then we know by soundness that $\nvdash \phi$, hence
$\invis^\phi , w \nVdash \phi$ for some $w$ extending $\varnothing$.
So it suffices to show that $|\invis^\phi| \in O(EXP2(|\phi|))$.

Note that $\invis^\phi \subseteq
\powerset(\Sigma(\delta(\phi),\phi))$.  We shall show that
$|\Sigma(\delta(\phi),\phi)| \in O(EXP(|\phi|))$, then we would know
that $|\powerset(\Sigma(\delta(\phi),\phi))| \in O(EXP2(|\phi|))$,
which would suffice to show the result.

First observe that $|\delta(\phi)| \in O(EXP(|\phi|))$.  This is
because in a worst case scenario, $\phi$ is constantly branching into
$\psi \to \chi$ formulae, and adding two new agents every time it
branches.  However, it cannot have more than $3^{|\phi|}$ agents even
in this worst case scenario, so we know that $|\delta(\phi)| \in
O(EXP(|\phi|))$.

 Since every non-branching step (ie. every time $\Sigma$ processes a
 formula not of the form $\psi \to \chi$) of
$\Sigma(\delta(\phi), \zeta)$ introduces at worst $O(|\delta(\phi)|) \in O(EXP(|\phi|))$
many formulae, we may again do a worst-case analysis.  In a worst case
scenario $\Sigma(\delta(\phi),\phi)$ must branch $O(EXP({|\phi|}))$
times and each time perform a $O(EXP(|\phi|))$ operation.  Even in
this worst case scenario, the complexity is still $O(EXP(|\phi|))$,
which is as we claimed.
\end{proof}

\begin{theorem}[\textsc{EviL} Decidability]\label{evil-decidability}\ 
\textsc{EviL} is decidable and 
the time complexity of the decision problem for \textsc{EviL} is
bounded above by $O(EXP3)$
\end{theorem}
\begin{proof}
 We know
that $\Cross^\phi \in \powerset (\powerset
(\Sigma(\delta(\phi),\phi)))$,
We know that
 $\phi$ is not a tautology of \textsc{EviL} if and only if there is a
 suitable \textsc{EviL} witnessing Kripke structure in $\powerset
 (\powerset (\Sigma(\delta(\phi),\phi)))$, defined in the manner of $\Cross$.
So a decision procedure to check if $\phi$ is an \textsc{EviL}
tautology 
is to check every member of
$\powerset (\powerset (\Sigma(\delta(\phi),\phi)))$ to see it gives
rise to an \textsc{EviL} model with some world which disproves
$\phi$. Since, as we saw in the proof of Theorem \ref{small-model-property}, we know
that $|\Sigma(\delta(\phi),\phi)| = O(EXP(|\phi|))$, this procedure takes $O(EXP3(|\phi|))$ many steps to complete.
\end{proof}

We shall now move on to showing how we may
recover a concrete \textsc{EviL} model from a finite \textsc{EviL}
Kripke structure.  The above results shall ensure completeness of
\textsc{EviL} for its intended semantics.  Before proceeding 
we shall first need to introduce the concept of a \emph{island}.

%%% Local Variables: 
%%% mode: latex
%%% TeX-master: "evil_philosophy"
%%% End: 
