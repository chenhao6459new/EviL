In this section, we define what it means for a model to be
\emph{partly \textsc{EviL}}.  We should note that partly \textsc{EviL}
models are exactly the same as \textsc{EviL} models as defined in Definition
  \ref{evil-kripke-structures} from \S\ref{kripke}, only property
  \ref{pVII} has been weakened to \ref{ppVII} and \ref{ppIX}.

\begin{definition}[Partly \textsc{EviL}]\label{partly-evil}
A Kripke structure $\mathbb{M}=\langle W, R, \sqsubseteq, \sqsupseteq, V, P \rangle$ is called \textbf{partly
  \textsc{EviL}} whenever it makes true the following properties, for
all agents $\{X,Y\} \subseteq \mathcal{A}$:
  \begin{enumerate}[label=\textup{(\emph{\Roman*})$'$}, topsep=0.0in, parsep=0.075in]
    \item\label{ppI} $\sqsubseteq_X$ is reflexive
    \item \label{pptrans} $\sqsubseteq_X$ is transitive 
    \item \label{ppantisym} $\sqsubseteq$ is a partial order
    \item \label{ppreverse} $w \sqsubseteq_X v$ if and only if $v
    \sqsupseteq_X w$
    \item \label{islandiff} If $w \sqsubseteq_X v$ then ($w \in V (p)$ if and only if $v
    \in V (p)$)
    \item \label{ppV} $(R_X \circ \sqsubseteq_X) \subseteq R_X$
    \item \label{ppVI} 
    $(\sqsubseteq_Y \circ R_X) \subseteq R_X$ and \\
    $(\sqsupseteq_Y \circ R_X) \subseteq R_X$
    \item\label{ppVII} If $w \in P_X$ then $w
    R_X w$
    \item\label{ppIX} If $w \in P_X$ and $w \sqsupseteq_X v$ then $v
    \in P_X$
  \end{enumerate}
\end{definition}

Note that it is elementary that every \textsc{EviL} structure is
partly \textsc{EviL} as well.

These properties are exactly the properties enforced
by the axioms of \textsc{EviL} given in Table \ref{table:axioms} in
\S\ref{evil-axioms}. We can see this by observing the following
theorem:

\begin{definition}\label{pEviL-Vdash}
We shall write
\[ \Gamma \Vdash_{\textup{p\textsc{EviL}}} \phi \]
to mean that for all partly \textsc{EviL} Kripke structures
$\mathbb{M} = \langle W, R, \sqsubseteq, \sqsupseteq, V, P \rangle$,
for all worlds $w \in W$ if $\mathbb{M},w \Vdash \Gamma$ then $\mathbb{M} \Vdash \phi$.
\end{definition}

\begin{theorem}[Partly \textsc{EviL} Strong Soundness and
  Completeness]\label{partly-evil-completeness}

$$\Gamma \vdash_{\textsc{EviL}} \phi\textup{ if and only if }\Gamma
\Vdash_{\textup{p\textsc{EviL}}} \phi$$

\end{theorem}
\begin{proof}
The left to right direction, \emph{soundness}, is trivial (one should
use induction).  So we shall focus on the right to left direction; 
to do this we shall consider the contrapositive.  We shall make heavy
use of \emph{correspondence} theory, namely the \emph{Sahlqvist
  Correspondence Theorem} \cite[Theorem 4.42,
pg. 212]{blackburn_modal_2001}.  We note that the axioms
\eqref{reflAx}, \eqref{transAx}, \eqref{downConceive}, \eqref{islandDown},
\eqref{islandUp}, \eqref{soundness}, \eqref{downsound},
\eqref{reverseAx1} and
\eqref{reverseAx2} are all
\emph{Sahlqvist formulae}.  When we say that a particular fact corresponds to an
axiom, we mean that from the Sahlqvist correspondence theorem and
first order logic one may be employed to show the fact in question.

So assume $\Gamma \nvdash \phi$, 
we must show that $\Gamma \nVdash \phi$.  
To see this, we carry out the canonical model construction as
described in \cite[chapter 4,
pgs. 198--422]{blackburn_modal_2001}\footnote{It is important to note
  that the results obtained in \cite{blackburn_modal_2001} are technically for any \emph{normal} modal logic,
  but they may be generalized to non-normal logics such as the
  \textsc{EviL} logic under consideration.}.  Let $\mathcal{E}$ be the set of maximally
  consistent sets of formulae for \textsc{EviL}. Define the
  \emph{canonical model}
  $$\mathscr{E} := 
\langle \mathcal{E}, R, \sqsubseteq, \sqsupseteq, V, P \rangle$$
where, for all $\{w,v\} \subseteq \mathcal{E}$:
\begin{bul}
  \item $w R_X v$ if and only if $\{ \phi\ |\ \Nec_X \phi \in w\}
    \subseteq v$
  \item $w \sqsubseteq_X v$ if and only if $\{ \phi\ |\ \BP_X \phi \in w\}
    \subseteq v$
  \item $w \sqsupseteq_X v$ if and only if $\{ \phi\ |\ \BM_X \phi \in w\}
    \subseteq v$
  \item $V(p) := \{ w \ |\ p \in w\}$
  \item $P_X := \{ w \ |\ \PP_X \in w \}$
\end{bul}

We know from the \emph{Lindenbaum Lemma} that $\Gamma$ may be extended
to some maximally consistent $\gamma$ such that $\Gamma \subseteq
\gamma$, $\gamma \in \mathcal{E}$ and $\phi \nin
\gamma$ \cite[Lemma 4.17,
pg. 199]{blackburn_modal_2001}.  By the \emph{Truth Lemma} we may establish
$\mathscr{E}, \gamma \nVdash \phi$ and $\mathscr{E}, \gamma \Vdash \Gamma$ \cite[Lemma 4.21,
pgs. 201]{blackburn_modal_2001}.  So it suffices to establish
that $\mathscr{E}$ is partly \textsc{EviL}, by establishing that it
satisfies the properties given in Definition \ref{partly-evil}.

  \begin{description}
    \item[\ref{ppI}] ``$\sqsubseteq_X$ is reflexive''
      corresponds to axiom \eqref{reflAx}.
    \item[\ref{pptrans}] ``$\sqsubseteq_X$ is transitive''
      corresponds to axiom \eqref{transAx}.
    \item[\ref{ppreverse}] ``$\sqsubseteq_X$ is the
      reverse $\sqsupseteq_X$'' corresponds to axioms \eqref{reverseAx1} and
      \eqref{reverseAx2}.
    \item[\ref{islandiff}] Assume $w \sqsubseteq_X v$, we shall show
      that 
$$w \in V (p)\textup{ if and only if }v \in V (p)$$
     Now assume that $w \in V(p)$, then $\mathscr{E}, w \Vdash p$.
      By axiom \eqref{letterAx2} and the Truth Lemma we have that $\BP_X
      p \in w$, whence $p \in v$ by definition.  The other direction
      is similar, however it uses axiom \eqref{letterAx1} instead.
    \item[\ref{ppV}] 
The assertion
\begin{eqnarray*}
& (R_X \circ \sqsubseteq_X) \subseteq
    R_X 
% \\
% & \& \\
% & (R_X \circ \sqsupseteq_X) \subseteq
%     R_X 
\end{eqnarray*}
corresponds to axiom \eqref{downConceive} 
(noting that one should reason given \ref{ppreverse}). 
% This is because axiom \eqref{downConceive} is a Sahlqvist
% formula and together with \ref{ppreverse} it codes for exactly this assertion.

    \item[\ref{ppIX}] ``If $w \in P_X$ and $w \sqsupseteq_X
      v$ then $v \in P_X$'' corresponds to axiom \eqref{downsound}.

     \item[\ref{ppantisym}] We have deferred the proof that
      $\sqsubseteq$ is a partial order, since it depends on the above
      results.  We know that it is reflexive and transitive by
      \ref{ppI} and \ref{pptrans}.  All that is left is to show that
      it is anti-symmetric.  Assume that $w \sqsubseteq v$ and $v
      \sqsubseteq w$.  By the Truth Lemma it suffices to show that
      $\mathscr{E}, w \Vdash \phi$ if and only if $\mathscr{E}, v
      \Vdash \phi$, since then $\phi \in w$ if and only if $\phi \in v$.  We shall show this by induction on $\phi$.
      \begin{description}
        \item[$p \in \Phi$ --]  We have this step from the assumption
          that $w \sqsubseteq v$ and \ref{islandiff}.
        \item[$\bot$ --]  This case is trivially true.
        \item[$\BP_X \phi$ and $\BM_X\phi$ --]  These steps follows from
          \ref{pptrans}, \ref{ppreverse} and the assumption.
        \item[$\Box_X \phi$ --]  This case follows from \ref{ppV} and
          the assumption.
        \item[$\PP_X$ --] This step follows from \ref{ppIX} and the assumption.
        \item[$\phi \to \psi$ --]  The final step follows trivially from the inductive hypothesis.
      \end{description}

    %   Assume that $x \sqsubseteq_X 

% First assume that $w R_X \circ \sqsubseteq_X v$, we must show that $w
% R_X v$.  First observe that we have that there is some
% $u$ such that $w \sqsubseteq_X u$, and $u R_X v$.
%   Now suppose towards a contradiction that $\neg w R_X v$, then there
%   must be some $\Nec_X \phi \in w$ where $\phi \nin v$.  We know
%   that since $u R_X v$ that $\Pos_X \neg \phi \in u$ by the Truth lemma.  However, from axiom
%   \eqref{downConceive} we then have that $\Pos_X \neg \phi \in w$, which
%   violates that $w$ is maximally consistent. $\lightning$

% Now assume that $w R_X v$ and $w \sqsupseteq_X u$.  We must show that
% $u R_X v$.  But we know from $w \sqsupseteq_X u $ and \ref{ppreverse}
% that $u \sqsubseteq_X w$, and with $w R_X v$ we have that $u R_X \circ
% \sqsubseteq_X v$, hence by the above we have $u R_X v$, as desired.


    \item[\ref{ppVI}] Given \ref{ppreverse}, the fact that
    \begin{eqnarray*}
& (\sqsubseteq_Y \circ R_X) \subseteq R_X \\
& \& \\
&  (\sqsupseteq_Y \circ R_X) \subseteq R_X
\end{eqnarray*}
corresponds to axioms \eqref{islandDown} and \eqref{islandUp}.

    \item[\ref{ppVII}] ``If $w \in P (X)$ then $w R_X w$'' corresponds
      to axiom \eqref{soundness}.

  \end{description}

\end{proof}
%%% Local Variables: 
%%% mode: latex
%%% TeX-master: "evil_philosophy"
%%% End: 
