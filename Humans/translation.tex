In this section, we turn to showing that every finite \textsc{EviL}
Kripke structure $\mathbb{M}$ has a corresponding \textsc{EviL} model
$\ipent$ which is an (almost)-homomorphic projection\footnote{Note
  that we shall not provide a formal definition of
  what it means for a map to be (almost)-homomorphic, since we
  consider this concept more intuitive than formal.  Intuitively, two objects
  are \emph{(almost)-homomorphic} when they are homomorphic for all
  intents and purposes.}.  Assuming that $\Phi$ is infinite and $\Psi
\subseteq_{\omega} \Phi$, then we shall show that $\mathbb{M}$
and $\ipent$ agree on the language $\mathcal{L}(\Psi,\mathcal{A})$.  
The method of the proof of this correspondence generalizes the
 elementary argument presented in Proposition
\ref{translation-sketch} from \S\ref{sketch}.  From this
correspondence, we shall obtain a weak completeness theorem for
\textsc{EviL} and its intended semantics.

Recall that in the proof of Proposition \ref{translation-sketch}
 we assumed an infinite store of unused letters, and assigned 
them to worlds in order to control the accessibility in the 
\textsc{EviL} model we constructed.  This was embodied by a function
$p : W \rightarrowtail \Phi \bs L(\phi)$; for each world $w$, $p_w$
was the \emph{name} we assigned to it.  In our construction here, we
shall extend this metaphor, using a generic finite set
$\Psi\subseteq_\omega \Phi$. 

Recall that among the three principle ways we described for thinking
about think about islands, one way to think of $\lcorners w \rcorners$ 
was as $w$'s extended family.  So along with 
\emph{personal names}, we shall also want to assign family 
names or \emph{surnames}.

With these above considerations in mind, we offer the following definition:

\begin{mydef}
Assume the set of letters $\Phi$ is infinite, and fix a finite $\Psi \subseteq_\omega
\Phi$, a finite \textsc{EviL} Kripke
model $\mathbb{M}$
\begin{bul}
\item Let $\Psi$ be a finite set of proposition letters.
% as in the proof of Proposition \ref{translation-sketch} from \S\ref{sketch}
\item Let $$\ang := \{\{w\}, \lcorners w\rcorners \ |\ w
  \in W^{\mathbb{M}}\}$$
That is, $\ang$ is the set of worlds and islands.
\item Let $p: \ang \rightarrowtail \Phi \bs \Psi$ be an
  injection, assigning names to worlds and surnames to
  islands\footnote{Subsequently, we shall abbreviate $p(\{w\})$ as $p_w$ and
    $p(\lcorners w \rcorners)$ as $p_{\lcorners w \rcorners}$}.
\item Let $\kl : W^{\mathbb{M}} \to \powerset \Phi \times
  (\powerset(\mathcal{L}_0(\Phi)))^\mathcal{A}$ be defined such that:
\[ \kl(w) := (\kl_1(w),\kl_2(w)) \]
Where:
\begin{bul}
  \item $\kl_1 : W^{\mathbb{M}} \to \powerset \Phi$ is defined to be:
\[ \kl_1(w) := \{q \in \Psi \ |\ \mathbb{M},w \Vdash q\} \cup
\{p_{\lcorners w\rcorners }\} \]
We may understand $\kl_1$ as providing a propositional valuation to
worlds in $W^{\mathbb{M}}$
\item $\kl_2 : W^{\mathbb{M}} \to \powerset(\mathcal{L}_0(\Phi))$ is defined to be:
\[\kl_2(w) := \prod_{X \in \mathcal{A}} \kl_{2A}(w,X) \cup
\kl_{2B}(w,X) \]
Where:
\begin{bul}
  \item $\kl_{2A} : W^{\mathbb{M}} \times \mathcal{A} \to
    \powerset(\mathcal{L}_0(\Phi))$ is defined to be:
\[ \kl_{2A}(w,X) :=\{ \neg p_{\lcorners v
  \rcorners} \ |\ \neg w R_X v\} \]
 \item $\kl_{2B} : W^{\mathbb{M}} \times \mathcal{A} \to
   \powerset(\mathcal{L}_0(\Phi))$ is defined to be:
\[\kl_{2B}(w,X) := \{ \bot \to p_{v}\ |\ w \sqsupseteq_X v\} \]
\end{bul}
We may understand $\kl_2$ as providing, for each agent, a
corresponding set of propositional formulae.  These formulae constitute their set of basic beliefs, as we originally introduced in
\S\ref{sketch}.

$\kl_{2A}$ and $\kl_{2B}$ each constitute a component that goes into
the basic belief set we assign to a particular agent.
\end{bul}

\item Let $\ipent := \kl[W]$

\end{bul}
\end{mydef}

Certain remarks must be made regarding the above definition.  

For one, note that we are ensured by the axiom of choice that 
$p_w$ is well defined, since by hypothesis we have that 
$W$ is finite, whence
$\powerset W$ is finite and since $\Lambda \subseteq \powerset W$ 
we know that $\Lambda$ is finite as well.  
Since we know that $\Phi$ is infinite then 
$\Phi \bs \Psi$ is infinite as well, and there always
exists an embedding of a finite set into an infinite set.

To be completely explicit about our intentions, $\ipent$ is an 
\textsc{EviL} model we are constructing which shall preserve 
the truth of $\phi$ for all of the worlds in $\mathbb{M}$.  
Our goal is that $\ipent$ should
be an \emph{(almost)-homomorphic projection of
  $\mathbb{M}$ under $\kl$} with respect to a language
$\mathcal{L}(L,\mathcal{A})$, where $L$ is a finite set of letters.  
This is precisely why we have set $\ipent$ to
be the image of $W$ under $\kl$. Permit us to explain what 
``almost homomorphic'' means exactly.

% First, observe that  
% Since the worlds in \textsc{EviL} models are pairs, $\kl$ is broken 
% up into  two operations $\kl_1$ and $\kl_2$ that give each 
% component of each pair.

% The two operations $\kl_1$ and $\kl_2$ deserve some extra motivation,
% since they may be challenging to understand. To see what is going on, 
Recall the definition of $\mho^{\ipent}$ from Definition
\ref{omega-translation} from \S\ref{kripke}. This defines
$\sqsubseteq^{\ipent}$, $\sqsupseteq^{\ipent}$, and $R^{\ipent}$.  To ensure
that $\ipent$ is (almost)-homomorphic to $\mathbb{M}$, 
we shall want to enforce the following relationships:

\begin{align}
q \in \Psi \Longrightarrow (\mathbb{M},w\Vdash q \iff &
\ipent,\kl(w)\VDash q) \label{lettersforce}\\
\mathbb{M},w\Vdash \PP_X \iff & \ipent,\kl(w)\VDash \PP_X \label{Pforce} \\
w \sqsubseteq^\mathbb{M}_X v \iff &  \kl(w) \sqsubseteq^{\ipent}_X
\kl(v) \label{subforce}\\
w \sqsupseteq^\mathbb{M}_X v \iff & \kl(w) \sqsupseteq^{\ipent}_X
\kl(v) \label{supforce}\\
v \in \lcorners w \rcorners^\mathbb{M} \iff &  \kl(v) \in \lcorners \kl(w) \rcorners^{\ipent} \label{islandforce}\\
w R^\mathbb{M}_X v \iff & \kl(w) R^{\ipent}_X
\kl(v) \label{relforce}
\end{align}

So in order for $\ipent$ to ``solve'' the above equations, we have
various logical constraints on our definitions,  which we have used to 
determined the design choices we have made.  We shall show that
$\ipent$ solves the above equations in Lemma \ref{ipentsolution}.

Before we go ahead and prove results about $\ipent$, we shall try to
brush up certain natural questions one may naturally ask about $\ipent$.
\begin{bul}
\item   \emph{Why does $\kl_1(w)$
encode $w$'s surname but not her full name?  That is, why is it that
$p_{\lcorners w\rcorners} \in
\kl_1(w)$ but $p_{w} \nin
\kl_1(w)$?}

Note that in our construction of $\ipent$ we have been trying to enforce
that \eqref{subforce}, \eqref{supforce} and \eqref{islandforce}.
From definition
\ref{omega-translation} from \S\ref{kripke} we know that if $\kl(w) \sqsubseteq^{\ipent}_X \kl(v)$ then
$\kl_1(w) = \kl_1(v)$. Hence we must define $\kl_1$ in such a manner
where if $p_w \in \kl_1(w)$ then $p_w \in \kl_1(v)$.  In fact, since
we are enforcing \eqref{islandforce}, then we know that we cannot encode any information in $\kl_1(w)$ without
putting it into $\kl_1(v)$ for any $v \in \lcorners
w\rcorners^\mathbb{M}$.  However, knowing that we intend to
preserve columns in our construction, we may safely encode 
information about column membership in $\kl_1$, as we have done.
\item \emph{Why does $\kl_2(w)$
encode $\neg p_{\lcorners v \rcorners}$, that is the negation of $v$'s
surname, when $\neg w R_X v$, as opposed to her full name?}  

Recall that we want to enforce that \eqref{relforce}.  We want to make
sure that $\kl(w)$ can ``see'' $\kl(v)$ in all and only those
situations when it is supposed to.  We accomplish this by encoding
surname information into $\kl_1(v)$, and ``blacklisting'' certain
surnames in $\kl_2(w)$ we do not want $w$ to ``see'' using
$R^{\ipent}_X$.  Here we are very consciously exploiting the Lemma
\ref{island}\ref{islandR2}, which asserts if one member of an island is
not accessible to $w$ then nobody on that island is.

\item \emph{Why does $\kl_2(w)$ encode the ``vacuous'' information that $\bot \to p_{v}$ when $w \sqsupseteq_X v$?}
In order to enforce \eqref{subforce} and \eqref{supforce}, we need to
encode information regarding $\sqsubseteq_X^\mathbb{M}$ and $\sqsupseteq_X^\mathbb{M}$
somewhere.  We cannot encode this information
in $\kl_1$, for the reason that it can only safely encode information
at the island level using surnames.  
Hence we must encode this information in $\kl_2$; 
it is for this reason that we have chosen to include 
$\bot \to p_{v} \in \kl_{2B}(w,X)$.

However, we do not want the information we encode in $\kl_{2B}(w,X)$ to
interfere with $R^{\ipent}_X$, so one way to ensure that ``harmless''
information is encoded is to use tautologies, as we have done.
\end{bul}

Hopefully the reader has some intuition about the engineering choices
we made in the construction of $\ipent$.  We now turn to proving
that $\ipent$ satisfies our design criteria.

\begin{lemma}\label{ipentsolution}
Provided that $\mathbb{M}$ is \textsc{EviL}, our definition of
$\ipent$ suffices \eqref{lettersforce} through \eqref{relforce}.
\end{lemma}
\begin{proof} \ 
  \begin{bul}
    \item \eqref{lettersforce} 
\begin{align*}
q \in \Psi \Longrightarrow (\mathbb{M},w\Vdash q \iff &
\ipent,\kl(w)\VDash q)
\end{align*}
Let $q \in \Psi $.   We have two directions we must reason:
\begin{description}
\item[$\Longrightarrow$]
First assume that $\mathbb{M},w\Vdash q$.  We know that 
\begin{align*}
\ipent,\kl(w)\VDash q & \iff q \in \kl_1(w) \\
& \iff q \in \{q \in \Psi \ |\ \mathbb{M},w \Vdash q\} \cup
\{p_{\lcorners w\rcorners }\}
\end{align*}
Hence $\ipent,\kl(w)\VDash q$ as desired.

\item[$\Longleftarrow$]
Assume that $\ipent,\kl(w)\VDash q$, we to show $\mathbb{M},w\Vdash
q$. By our assumption we have either $q \in \{q \in
L \ |\ \mathbb{M},w \Vdash q\}$ or $q \in \{p_{\lcorners
  w\rcorners }\}$.  In the former case we are done, and the latter
case is impossible since $p_{\lcorners w \rcorners} \in \Phi \bs \Psi $ 
by definition, hence it is impossible for $q \in
\{p_{\lcorners w\rcorners }\}$ by hypothesis.
\end{description}

\item \eqref{Pforce} 
\begin{align*}
\mathbb{M},w\Vdash \PP_X \iff & \ipent,\kl(w)\VDash \PP_X
\end{align*}
Since $\mathbb{M}$ is \textsc{EviL}, and
      $\ipent, \kl(w) \VDash \PP_X$ if and only if $\kl(w)
      R^{\ipent}_X \kl(w)$, by virtue of property 
      \ref{pVI} of \textsc{EviL} Kripke models it suffices 
     to prove \eqref{relforce} below.


\item \eqref{subforce}
\begin{align*}
w \sqsubseteq^\mathbb{M}_X v \iff &  \kl(w) \sqsubseteq^{\ipent}_X
\kl(v)
\end{align*}  
We have two directions to show:
\begin{description}
\item[$\Longrightarrow$] Assume that $w \sqsubseteq^\mathbb{M}_X v$.
  To ensure $\kl(w) \sqsubseteq^{\ipent}_X \kl(v)$ we need to ensure
  two things:
\begin{myroman}
\item $\kl_1(w) = \kl_1(v)$ --  In order for this to be the case, we
  must have: 
\[ \underbrace{\{q \in \Psi \ |\ \mathbb{M},w \Vdash q\}}_A \cup
\underbrace{\{p_{\lcorners w\rcorners }\}}_B = \underbrace{\{q \in \Psi \ |\ \mathbb{M},v \Vdash q\}}_C \cup
\underbrace{\{p_{\lcorners v\rcorners }\}}_D\]
Note that by hypothesis, $w$ and $v$ are on the same island, which
means that $B = D$.   Since if two worlds in an \textsc{Evil} model
are on the same island,
%  (ie. $w \sqsubseteq_X^\mathbb{M} v
% \Longrightarrow \lcorners w \rcorners = \lcorners v \rcorners$) 
then by the Island Lemma they make the same proposition letters true, hence $A =
C$, which suffices.
\item $(\kl_2(w))_X \subseteq  (\kl_2(v))_X$ --  Since $(\kl_2(u))_X =
  \kl_{2A}(u,X) \cup \kl_{2B}(u,X)$, it suffices to show that
  $\kl_{2A}(w,X) \subseteq \kl_{2A}(v,X)$ and $\kl_{2B}(w,X) \subseteq
  \kl_{2B}(v,X)$:
\begin{bul}
\item $\kl_{2A}(w,X) \subseteq \kl_{2A}(v,X)$ --  Assume that $x \in
  \kl_{2A}(w,X)$.  Then $x = \neg p_{\lcorners u\rcorners}$ for some $u \in W$
  where $\neg w R_X^{\mathbb{M}} u$. It suffices to show that $\neg v R_X^{\mathbb{M}} u$.  

Suppose towards a contradiction that $v R_X^{\mathbb{M}} u$, 
then by hypothesis we have that $w
  R_X^{\mathbb{M}} \circ \sqsubseteq_X^{\mathbb{M}} u$.  However, we know that since
  $\mathbb{M}$ is \textsc{EviL} then by \ref{pV} we have that $R_X^{\mathbb{M}}
  \circ \sqsubseteq_X^{\mathbb{M}} \subseteq R_X^{\mathbb{M}}$, which
  means that $w R_X^{\mathbb{M}} u$ after all. $\lightning$ 
\item $\kl_{2B}(w,X) \subseteq \kl_{2B}(v,X)$ --  Assume that $x \in
  \kl_{2B}(w,X)$, then $x = \bot \to p_u$ for some $u$ such that $u
  \sqsubseteq_X^{\mathbb{M}} w$.  Then by transitivity we have that $u
  \sqsubseteq_X^{\mathbb{M}} v$, which means that $\bot \to p_u \in \kl_{2B}(v,X)$  as desired.
\end{bul}  
\end{myroman}

\item[$\Longleftarrow$]  Assume that $\kl(w) \sqsubseteq_X^{\ipent}
  \kl(v)$.  We know that since $\mathbb{M}$ is \textsc{EviL} then
  $\sqsubseteq^\mathbb{M}_X$ is reflexive, so $w \sqsubseteq_X^\mathbb{M} w$, whence $\bot \to
  p_w \in \kl_{2B}(w)$.  Thus $\bot \to p_w \in (\kl_2(v))_X$, which means
  that either $\bot \to p_w \in \kl_{2A}(v,X)$ or $\bot \to p_w \in
  \kl_{2B}(v,X)$.  We can see that $\bot \to p_w \neq \neg
  p_{\lcorners u\rcorners}$ for all $u$ since these formulae are of different forms, so it must be that $\bot \to p_w \in \kl_{2B}(v)$.  This means that $w
  \sqsubseteq_X^\mathbb{M} v$, as desired.
\end{description}
\item \eqref{supforce} 
\begin{align*}
w \sqsupseteq^\mathbb{M}_X v \iff & 
\kl(w) \sqsupseteq^{\ipent}_X \kl(v) \\
\end{align*}

This
  follows from \eqref{subforce} and the fact that both $\mathbb{M}$ and $\ipent$ are
  \textsc{EviL}, hence $x \sqsubseteq_X y \iff y \sqsupseteq_X x$ for
  both structures.

\item \eqref{islandforce}  
\begin{align*}
v \in \lcorners w \rcorners^\mathbb{M} \iff &  \kl(v) \in \lcorners \kl(w) \rcorners^{\ipent}
\end{align*}
The fact that islands in both structures
  correspond follows from the correspondences between $\sqsubseteq_X$ and
  $\sqsupseteq_X$, as we already saw in \eqref{subforce} and \eqref{supforce}.
\item\eqref{relforce} 
\begin{align*}
w R^\mathbb{M}_X v \iff & \kl(w) R^{\ipent}_X \kl(v)
\end{align*}
\begin{description}
\item[$\Longrightarrow$]
First assume that $w R^\mathbb{M}_X v$, we want to show $\kl(w) R^{\ipent}_X
\kl(v)$.  
This means that we must show $\kl_1(v) \models
(\kl_2(w))_X$.  
Since $(\kl_2(w))_X = \kl_{2A}(w,X) \cup
\kl_{2B}(w,X)$, we have two steps:
\begin{description}
  \item[$\kl_1(v) \models \kl_{2A}(w,X)$ --]  Assume that $\kl_1(v)
    \nmodels \kl_{2A}(w,X)$, then it must be that there is some $u\in
    W^\mathbb{M}$ where $p_{\lcorners u\rcorners} \in
    \kl_1(u)$ and $\neg p_{\lcorners u\rcorners} \in \kl_{2A}(w)$,
    which means that $\neg w R^{\mathbb{M}}_X u$.
    Since $p_{\lcorners u\rcorners} \nin L(\Phi)$ it must be that
    $p_{\lcorners u\rcorners} = p_{\lcorners v\rcorners}$, hence $\lcorners u \rcorners = \lcorners v \rcorners$. Then by the
     Island Lemma we have $\neg w R^{\mathbb{M}}_X v$ after
    all. $\lightning$
  \item[$\kl_1(v) \models \kl_{2B}(w,X)$ --]  Simply note that
    everything in $\kl_{2B}(w,X)$ is a tautology, by construction, 
    so this step follows vacuously.
\end{description}
\item[$\Longleftarrow$]  Assume $\kl(w) R^{\ipent}_X \kl(v)$, in other words
  $\kl_1(v) \models (\kl(w))_X$.  We shall show $w R^\mathbb{M}_X v$.
  So suppose to the contrary that $\neg w R^\mathbb{M}_X v$, then
  $\neg p_{\lcorners v\rcorners} \in \kl_{2A}(w)$.  However we know
  that $p_{\lcorners
    v\rcorners} \in \kl_1(v)$, hence $\kl_1 \models p_{\lcorners
    v\rcorners}$, which means that $\kl_1(v) \nmodels (\kl(w))_X$,
  which contradicts our assumption. $\lightning$
\end{description}
\end{bul}
\end{proof}

Having established that $\ipent$ is indeed (almost)-homomorphic to
$\mathbb{M}$, we may use this to show that $\mathbb{M}$ and $\ipent$
are logically the same over $\mathcal{L}(\Psi, \mathcal{A})$.

\begin{lemma}[\textsc{EviL} Translation]\label{translation-lemma}
Let $\mathbb{M}$ be an \textsc{EviL} Kripke structure.  For any
formula $\phi \in \mathcal{L}(\Psi)$, and any $w \in W$, we have
\[ \mathbb{M},w \Vdash \phi \iff \ipent, \kl(w) \VDash \phi \]
\end{lemma}
\begin{proof}
Using induction, and Lemma \ref{ipentsolution}, the result follows
from the fact that $\ipent$ and $\mathbb{M}$ correspond in all of the
ways relevant to $\mathcal{L}(\Psi,\mathcal{A})$. 
\end{proof}

Hence, from the above, we may prove a central result of \textsc{EviL}:

\begin{theorem}[\textsc{EviL} Soundness and Weak Completeness]
\label{conc-evil-completeness}
\[ \vdash_{\textsc{EviL}} \phi \iff \VDash \phi \]
\end{theorem}
\begin{proof}
Soundness is trivial, so we shall only prove completeness.

Assume that $\nvdash \phi$. We know from Theorem
\ref{abst-finite-completeness} that there is a finite $\mathbb{M}$
such that $\mathbb{M},w \nVdash \phi$ for some $w \in W$.

Now let $\Psi = L(\phi)$, where $L(\phi)$ is the letters that occur in
$\phi$, just as we originally defined in the proof of Proposition \ref{translation-sketch} 
from \S\ref{sketch}.  Since 
$\phi \in \mathcal{L}(L(\phi),\mathcal{A})$, from lemma
\ref{translation-lemma} we have that $\ipent, \kl(w) \nVDash \phi$.
Then evidently $\ipent$ is our desired counter model, hence we have
the theorem.
\end{proof}

With this, we may conclude the proof of completeness of \textsc{EviL}.
%%% Local Variables: 
%%% mode: latex
%%% TeX-master: "evil_philosophy"
%%% End: 
